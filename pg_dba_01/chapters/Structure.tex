\chapter{Install structure}\label{cha:INSTALLSTRUCT}
In this chapter we'll look at the PostgreSQL installation.
Whatever installation method you did, the PostgreSQL binaries are the same.
The packaged version comes with few extra utilities we'll take a look later as
those present some caveats to be aware.

The source installation puts all the binaries into the bin sub directory in the target location 
specified by the --prefix parameter.

The packaged install puts the binaries into a folder organised per major PostgreSQL version.

\textbf{e.g. /usr/lib/postgresql/9.3/bin/}

\section{The core binaries}
We'll look first at the core binaries like postgres or psql.

A separate section is dedicated to wrappers and the contributed modules.


\subsection{postgres}\index{postgres}
Is the database itself. It's possible to start it directly or using the pg\_ctl
utility.

The latter is the best way to control the instance except in case of the XID
wraparound failure\index{XID wraparound failure}. 
In this case the only way to run the instance is executing postgres in
single user mode.

For historical reason there's usually also the postmaster symbolic link
to postgres.


\subsection{pg\_ctl}\index{pg\_ctl}
\label{sub:PGCTL}
As mentioned before this is the utility for managing the PostgreSQL instance. 

It can start,stop, reload the postgres process.
It's also capable to send kill signals to the running instance. 

pg\_ctl accepts on the command line various options. The most important are the
-D or --pgdata= to specify the database storage area and the -m to specify the
shutdown mode. Check the section \ref{sec:SHUTDOWN_SEQ} for details. 

pg\_ctl also requires the action to perform on the instance.

The supported actions are

\begin{itemize}
 \item \textbf{init[db]} initialises a directory as PostgreSQL data area
 \item \textbf{start} starts a PostgreSQL instance
 \item \textbf{stop} shutdowns a PostgreSQL instance
 \item \textbf{reload} reloads the configuration's files
 \item \textbf{status} checks the PostgreSQL instance running status
 \item \textbf{promote} promotes a standby server 
 \item \textbf{kill} sends a custom signal to the running instance
\end{itemize}


\subsection{initdb}\index{initdb}
Is the binary which initialises the PostgreSQL data area. 
initdb requires an empty directory to initialise. 
Various options can be specified on the command line, like the character enconding or the collation order. 


\subsection{psql}\index{psql}
Is the PostgreSQL command line client. 
Despite his look very essential is one of the most flexible tools available to interact with the server.
As is part of the core distribution is always present.

\subsection{pg\_dump}\index{pg\_dump}
\label{sub:PGDUMP}
Is the binary dedicated to the backup. Generates consistent backups in various formats. The default is plain text. 
Supports the parallel dump implemented using the snapshot exports. 

The switch to set is -F followed by a letter to indicate the wanted output format.

\begin{itemize}
 \item \textbf{p} saves the sql statements to reconstruct the schema and/or data in plain text with no compression.
 \item \textbf{c} is the custom PostgreSQL format. Supports parallel restore, compression and object search.
 \item \textbf{d} the dump is saved in a directory. With this format is possible to dump in parallel.
 \item \textbf{t} saves the dump in the standard tar format.
 \end{itemize}

Please on't be confused by the pg\_dumpall \index{pg\_dumpall}. This does look more like a wrapper for pg\_dump rather a dedicated program.

As pg\_dumpall doesn't support all the pg\_dump features is still very useful to save the cluster wide objects like the users with the switch --globals-only.

For the usage check \ref{sec:PGDUMP}


\subsection{pg\_restore}\index{pg\_restore}
As the name suggests this is used to restore the database's dump.
It can read the backups generated in all formats by pg\_dump. 
The restore target can be a PostgreSQL connection or a file. 
If the backup's format is directory or custom then pg\_restore can run the data load and
index/key creations in multiple jobs.

\subsection{pg\_controldata}\index{pg\_controldata}\label{sub:PGCONTROLDATA}
The program query the pg\_control file where instance's vital informations are stored.
The pg\_control is one of the most important cluster's files.
With a corrupted pg\_control the instance cannot start. 

\subsection{pg\_resetxlog}\index{pg\_resetxlog}

If the WAL files or the get corrupted the instance cannot perform a crash recovery.
pg\_resetxlog can solve the problem and make the instance startable but keep in mind 
this must the last chance, after trying any other possible solution.

The reset removes the WAL files and creates a new pg\_control.
The XID are also restarted. 

The instance becomes startable at the cost of losing any reference between the transactions 
and the data files. All the physical data integrity is lost and
any attempt to run DML queries results in data corruption. 

The on line manual is absolutely clear on this point.

After running pg\_resetxlog the database must start without user access, 
the entire content must be dumped, the data directory must be dropped and recreated 
from scratch using initdb and then the dump file can be restored using psql or pg\_restore

\section{Wrappers and contributed modules}
In this section we'll take a brief look to the contributed binaries and the sql wrappers.

\subsection{create/drop binaries}
These binaries, createdb createlang createuser and dropdb droplang dropuser, are wrappers for the corresponding SQL functions. 
Each binary can create/drop a database an user or a procedural language.
The command line parameters are quite the same as psql for the connection part.

\subsection{clusterdb}\index{clusterdb}
Performs a database wide cluster on previously clustered tables. 
The binary can run on single tables specified on the command line.
The word cluster can be confusing as the PostgreSQL implementation is very peculiar.
Check the chapter \ref{cha:MAINTENANCE} for further readings.

\subsection{reindexdb}\index{reindexdb}
Performs a database wide reindex. 
It's possible to specify on the command line the target table or the target index.
In chapter \ref{cha:MAINTENANCE} we'll look deeply to the index maintenance,
how this can affect the performances.

\subsection{vacuumdb}\index{vacuumdb}
This binary is a wrapper for the VACUUM \index{VACUUM} SQL command.
The VACUUM is the most important maintenance task and is required every 2 billion transactions to avoid the XID wraparound failure \index{XID wraparound failure}
Similar to reindexdb and clusterdb, vacuumdb performs a database wide normal VACUUM.
It's possible to specify a target table on the command line, plus various options to have a VACUUM FULL \index{VACUUM FULL} or an ANALYZE \index{ANALYZE}.

\subsection{vacuumlo}\index{vacuumlo}
Despite the name this binary doesn't perform any vacuum, its purpose is to remove orphaned large
object from the pg\_largeobject. The pg\_largeobject is a system table where the database stores
the binary objects. Usually is used when the size of the large object is bigger than 1GB.
The theoretical limit for the large object is now 4 TB. Before the version 9.3 the limit were 2 GB.

\section{Debian's specific utilities}
The debian packaged install ships with some other, not official, utilities, mostly written in PERL.

\subsection{pg\_createcluster}\index{pg\_createcluster}
Creates a new PostgreSQL cluster naming the configuration's directory in /etc/postgresql after the
major version and the cluster's name. 
It's possible to specify the data directory and the initd options.

\subsection{pg\_dropcluster}\index{pg\_dropcluster}
Removes a PostgreSQL cluster created previously with pg\_createcluster. The cluster must be stopped before the drop.


\subsection{pg\_lscluster}\index{pg\_lscluster}
Lists the clusters created with pg\_createcluster.

\subsection{pg\_ctlcluster}\index{pg\_ctlcluster}
\label{sub:PGCTLDEB}
Controls the cluster in an almost similar way pg\_ctl does. 
Using this wrapper for the shutdown is not a good idea as there's no way to tell 
the script which shutdown mode to use. For further readings on the shutdown 
sequence check the section \ref{sec:SHUTDOWN_SEQ} 
By default pg\_ctlcluster performs a \textit{smart} shutdown mode.
Using the --force option the script first try a \textit{fast} shutdown mode.
If the database doesn't shutdown in a \textit{reasonable time} the script then try an \textit{immediate} shutdown.
If the the instance is still up the script then sends a \textbf{kill -9} on the postgres process.

