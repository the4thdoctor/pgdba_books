\chapter{A couple of things to know before start coding...}
\label{cha:COUPLETHINGS}
This chapter is to the developers approaching the PostgreSQL universe. PostgreSQL is a fantastic 
infrastructure for building powerful and applications. In order to use it at its best,  some things to 
consider. In particular some subtle caveats can make the difference between a magnificent success or a 
miserable failure. 

\section{SQL is your friend}
Recently the rise of the NOSQL engines, has shown more than ever how SQL is a fundamental requirement for 
managing the data efficiently. Shortcuts to the solution, like the ORMs sooner or later will show their 
limits. Despite the bad reputation the SQL language is very simple to understand, with few English keywords
and a powerful structured syntax which interacts perfectly with the regulated database layer. 
This simplicity have a cost. The language is parsed and converted into the database's structure causing 
sometimes misunderstanding between developer requests and the database results.\newline 

Mastering the SQL is a slow and difficult process and requires some sort of empathy with the DBMS. 
Asking for advice to the database administrator when building any design is a good idea to have a better 
understanding of what the database thinks. Having a few words with the DBA is a good idea in any case 
though.

\section{The design comes first}
One of the worst mistakes when building an application is to forget about the foundation, the 
database. With the rise of the ORM\footnote{Yes, I hate the ORMs} this is happening more frequently than it 
could be expected. Sometimes the database itself is considered just storage, a big mistake.\newline

The database design is either a complex and important and too much delicate to make it using by a dumb 
automatic tool. For example, using a generic abstraction layer will build access methods that will almost 
certainly ignore the PostgreSQL peculiar update strategy, resulting in bloat and general poor 
performance.\newline

It doesn't matter if the database is simple or the project is small. Nobody knows how successful could be 
a new idea. A robust design will make the project scale properly.

\section{Clean coding}
Writing decently formatted code is something any developer should do. This have a couple of immediate 
advantages. Improves the code readability when other developers need review it. Makes the code more 
manageable when, for example, is read months after it was written. This good practice forgets constantly 
to include the SQL. Is quite common to find long queries written all lowercase on one line with and the 
keywords used as identifier.\newline

Trying to read such queries is a nightmare. Often it takes more time in reformatting the queries rather 
doing the performance tuning. The following guidelines are a good reference for writing decent SQL and 
avoid a massive headache to the DBA.

\subsection{The identifier's name}
Any DBMS have its way of managing the identifiers. PostgreSQL converts all lowercase. This doesn't work very 
well with the camel case. It's still possible to mix upper and lower case letters enclosing the identifier 
name between double quotes. But that means the quoting should be put everywhere. Using the underscores 
instead of the camel case is simpler.

\subsection{Self explaining schema}
When a database structure becomes complex is very difficult to say what is what and how relates with the 
other objects. A design diagram or a data dictionary can help. But they can be outdated or not available. 
Using a simple notation to add to the relation's name will give an immediate outlook of the object's 
kind.\newline


\begin{table}[H]
\begin{tabular}{ll}
 \textbf{Object} & \textbf{Prefix}  \\
 \hline
 Table & t\_ \\
 View & v\_ \\
Btree Index & idx\_bt\_ \\
GiST Index & idx\_gst\_ \\
GIN Index & idx\_gin\_ \\
Unique index & u\_idx\_ \\
Primary key & pk\_ \\
Foreign key & fk\_ \\
Check & chk\_ \\
Unique key & uk\_ \\
Type & ty\_ \\
Sql function & fn\_sql\_ \\
PlPgsql function & fn\_plpg\_ \\
PlPython function & fn\_plpy\_ \\
PlPerl function & fn\_plpr\_ \\
Trigger & trg\_ \\
rule & rul\_ \\

\end{tabular}
\end{table}

A similar approach can be used for the column names, making the data type immediately recognisable.

\begin{table}[H]
\begin{tabular}{ll}
 \textbf{Type} & \textbf{Prefix}  \\
 \hline
 Character & c\_ \\
 Character varying & v\_ \\
Integer & i\_ \\
Text & t\_ \\
Bytea & by\_ \\
Numeric & n\_ \\
Timestamp & ts\_ \\
Date & d\_ \\
Double precision & dp\_ \\
Hstore & hs\_ \\
Custom data type & ty\_ \\

\end{tabular}
\end{table}



\subsection{Query formatting}
Having a properly formatted query helps to understand which objects are involved in the request and their 
relations. Even a simple query can be difficult to understand if badly written.


\begin{lstlisting}[style=pgsql]
select * from debitnoteshead a join debitnoteslines b on debnotid 
where a.datnot=b. datnot and b.deblin>1;
\end{lstlisting}

Let's list the query's issues.

\begin{itemize}
 \item using lowercase keywords it makes difficult to distinguish them from the identifiers
\item the wildcard * mask which fields  are really needed; returning all the fields consumes more bandwidth 
than required; it prevents the index only scans 
\item the meaningless aliases like \textit{a} and \textit{b} are confusing the query's logic
\item without proper indention the query logic cannot be understood
\end{itemize}

Despite existence of tools capable to prettify such queries, their usage doesn't solve the
root problem. Writing decently formatted SQL helps to create a mental map of what the query should removing 
as well the confusion when building the SQL. \newline

The following rules should be kept in mind constantly when writing SQL.

\begin{itemize}
 \item All SQL keywords should be in upper case
 \item All the identifiers and keywords should be grouped at same indention level and separated with a line 
break 
 \item In the SELECT list avoid the wildcard *
 \item Specify explicitly the join method in order to make it clear the query's logic
 \item Adopt meaningful aliases
\end{itemize}

This is the prettified query.

\begin{lstlisting}[style=pgsql]
SELECT
        productcode,
        noteid,
        datnot
FROM
        debitnoteshead head
        INNER JOIN
                debitnoteslines lines
                ON
                                head.debnotid=lines.debnotid
                        AND     head.datnot=lines.datnot
WHERE
               lines.deblin>1
;
\end{lstlisting}


\section{Get DBA advice}
\label{sec:GETDBA}
The database administration is weird. It's very difficult to explain what a DBA does. It's a job
where the statement ``failure is not an option" is the rule number zero. A DBA usually works in
antisocial hours, with a very tight time schedule.\newline

Despite the strange reputation, a database expert is an incredible resource for building up
efficient designs. Nowadays is very simple to set up a PostgreSQL cluster. Even with the default 
configuration the system is so efficient that under normal load doesn't show any problem. This could look 
like a fantastic feature but actually is a really bad thing. Any mistake at design level is hidden and when 
the problem appears is maybe too late to fix it.\newline

This final advice is probably the most important of the entire chapter. If you have a DBA don't be
shy. Ask for any suggestion, even if the solution seems obvious or if the design seems simple. The
database layer is a universe full of pitfalls where a small mistake can result in a very big 
problem.\newline

Of course if there's no DBA, that's bad. Never sail without a compass. Never start a database
project without an expert's advice. Somebody to look after of the most important part of the business, the
foundations.\newline

