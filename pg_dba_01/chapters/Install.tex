\chapter{Database installation}
\label{cha:DB_INSTALL}
This chapter will cover the install procedure, on Debian Gnu linux compiling from source and using the packaged install from the pgdg archive.
\index{pgdg}

\section{Install from source}
Installing from source, using the default configuration settings requires the root access as the default install location is in /usr/local/. 
To simplify things I've created a procedure with minimal need for root access.
This of course is still required but only for the os user creation and to install the dependencies. 

Before starting with the postgresql part, ask your sysadmin, or do it by yourself, to do the following

\begin{itemize}

 \item Create a postgres group and a postgres user
 \item Add the postgres user to the postgres group
 \item Install the packages
 \item build-essential
 \item libreadline6-dev
 \item zlib1g-dev
 
\end{itemize}


When everything is in place login as postgres user and download the source's tarball.

\textbf{
mkdir download
cd download
wget http://ftp.postgresql.org/pub/source/v9.3.4/postgresql-9.3.4.tar.bz2
}

Extract the tarbal with

\textit{tar xfj postgresql-9.3.4.tar.bz2} 
\textit{cd postgresql-9.3.4}

Using the configuration’s script option --prefix is possible to change 
the install directory to a custom location.
Assuming the postgres user have his home directory in /home/postgres, we'll put 
the install target into the bin directory organised per
mayor version. In this way  is possible to have multiple versions on the same 
box without hassle.

\textit{mkdir -p /home/postgres/bin/9.3} 
\textit{./configure --prefix=/home/postgres/bin/9.3}

The script will check all the dependencies and will generate the makefiles. 
Any error at configure time is usually explained good.

When everthing looks fine you can start the build process with the \textit{make} command.
The time required to compile depends from the box speed. On a laptop usually this doesn't require more than 30 minutes.

After the build is complete is a good idea to run the regression tests before installing.

To do it simply run \textit{make check}.

All the output is written in the directory src/test/regress/results. 

If everything is fine you can complete the installation with

\textit{make install}

After this, into the /home/postgres/bin/9.3 directory will appear with 4 subfolders 
\textbf{bin} \textbf{include} \textbf{lib} and \textbf{share}.

The \textbf{bin} folder is the place of the database binaries, like the client \textbf{psql} or the server \textbf{postgres}. 

The \textbf{include} folder is the place for the server's header files.

The \textbf{lib} folder is the location for all shared libraries and functions used by the database installation.

In the \textbf{share} folder you'll find the example files and the extension. 


\section{Packaged install}
\label{sec:DEBIAN_INSTALL}

The PostgreSQL Global Group mantains an apt repository to simplify the install on the GNU/Linux based on debian.

The supported Linux versions are listed on the wiki page http://wiki.postgresql.org/wiki/Apt and at moment are

\begin{itemize}
 \item Debian 6.0 (squeeze)
 \item Debian 7.0 (wheezy)
 \item Debian unstable (sid) 
 \item Ubuntu 10.04 (lucid)
 \item Ubuntu 12.04 (precise)
 \item Ubuntu 13.10 (saucy)
 \item Ubuntu 14.04 (trusty) 
\end{itemize}
The packages are available for amd64 and i386.

The available database versions are
\begin{itemize}
 \item PostgreSQL 8.4 
 \item PostgreSQL 9.0 
 \item PostgreSQL 9.1 
 \item PostgreSQL 9.2 
 \item PostgreSQL 9.3
\end{itemize}

Before starting you shall import the GPG key to validate the packages.

In a root shell simply run
\textit{wget --quiet -O - https://www.postgresql.org/media/keys/ACCC4CF8.asc | sudo apt-key add -}

Then add the file pgdg.list into the directory /etc/apt/sources.d/ with the following contents

\textbf{deb http://apt.postgresql.org/pub/repos/apt/ \textit{codename}-pgdg main}

Change the codename value accordingly with your distribuition. (e.g. wheezy) then you can run

\textit{apt-get update}

To install the full set of binaries simply run

apt-get install postgreql-9.3 postgreql-contrib-9.3 postgreql-client-9.3 

This will set up also a database cluster up and running.

