\chapter{Database installation}
\label{cha:DB_INSTALL}
In this chapter will see how to install PostgreSQL on Debian Gnu Linux. We'll take a look to 
two procedures, compiling from source and using the packages shipped by the pgdg 
\index{pgdg apt repository} apt repository.\newline 
There are advantages and disadvantages on both procedures. The compile from source offers a fine
grain control on all the aspects of the binaries configuration. Also doesn't have risks of
unwanted restarts when upgrading and it's possible to install and upgrade the binaries with normal
privileges.\newline

The packaged install is easier to manage when deploying the new binaries on the server, in
particular if there is a large number of installations to manage. The binary packages are
released shortly after a new update is released. Because the frequency for the minor
releases is not fixed, it could happen to have in place bugs affecting the production for months.
For example the bug causing the standby server crashing when the master found invalid pages during
a conventional vacuum, it was fixed almost immediately. Unfortunately the release with the fix
appeared after five months.\newline


\section{Install from source}
\label{sec:INSTSOURCE}
Using the configure script with the default settings requires the root access when installing.
That's because the permissions in the target location /usr/local don't allow the write for normal
users. This method adopts a different install location and requires the root access only for the os
user creation and the dependencies install. 
Before starting the PostgreSQL part ask your sysadmin to run the following commands.

\begin{itemize}

 \item useradd -d /home/postgres -s /bin/bash -m -U postgres
 \item passwd postgres 
 \item apt-get update
 \item apt-get install build-essential  libreadline6-dev  zlib1g-dev
\end{itemize}

Please note the second step will require inserting a new user password. Unless is a personal
test it's better to avoid obvious passwords like \textit{postgres}.\newline

In order to build the binaries we must download and extract the PostgreSQL's source tarball.

\begin{tinyverbatim}
mkdir ~/download
cd ~/download
wget https://ftp.postgresql.org/pub/source/v9.4.1/postgresql-9.4.1.tar.bz2
tar xfj postgresql-9.4.1.tar.bz2
cd postgresql-9.4
\end{tinyverbatim}


Using the configure script's option --prefix we'll point the install directory to a writable
location. We can also use a director named after the the major version's numbering. This will allow
us to have installed different PostgreSQL versions without problems.

\begin{tinyverbatim}
mkdir -p /home/postgres/bin/9.4
./configure --prefix=/home/postgres/bin/9.4
\end{tinyverbatim}

The configuration script will check all the dependencies and, if there's no error, will generate 
the makefiles. Then we can start the build simply running the command \textit{make}. The time
required for compiling is variable and depends from the system's power. If you have a multicore
processor the make -j option can improve significantly the build time. When the build is is complete
it's a good idea to to run the regression tests. Those tests are designed to find any regression or
malfunction before the binaries are installed.

\begin{tinyverbatim}
make
<very verbose output>

make check 

\end{tinyverbatim}

The test's results are written in the source's subdirectory src/test/regress/results. If
there's no error we can finalise the installation with the command make install.

\begin{tinyverbatim}
make install
\end{tinyverbatim}



\section{Packaged install}
\label{sec:DEBIAN_INSTALL}

The PostgreSQL Global Development Group manages a repository in order to facilitate the
installations on the Linux distributions based on the debian's packaging system. 

Currently the supported distributions are

\begin{itemize}
 \item Debian 6.0 (squeeze)
 \item Debian 7.0 (wheezy)
 \item Debian unstable (sid) 
 \item Ubuntu 10.04 (lucid)
 \item Ubuntu 12.04 (precise)
 \item Ubuntu 13.10 (saucy)
 \item Ubuntu 14.04 (trusty) 
 \item Ubuntu 14.10 (utopic) 
\end{itemize}

The PostgreSQL's versions available are 
\begin{itemize}
 \item PostgreSQL 8.4
 \item PostgreSQL 9.0 
 \item PostgreSQL 9.1 
 \item PostgreSQL 9.2 
 \item PostgreSQL 9.3
 \item PostgreSQL 9.4
\end{itemize}

The packages are available either for amd64 and i386 architecture.

The updated list is available on the the wiki page though.\newline
\href{http://wiki.postgresql.org/wiki/Apt}{http://wiki.postgresql.org/wiki/Apt}.\newline

All the installation steps require root privilege.Before starting configuring the repository it's also 
required to import the GPG key for the package validation.

In a root shell simply run
\begin{tinyverbatim}
apt-get install wget ca-certificates
wget --quiet -O - https://www.postgresql.org/media/keys/ACCC4CF8.asc | sudo apt-key add -
\end{tinyverbatim}

When the key is imported we'll create a file named pgdg.list into the directory /etc/apt/sources.d/ and
add the following row.

\begin{tinyverbatim}
deb http://apt.postgresql.org/pub/repos/apt/ {codename}-pgdg main
\end{tinyverbatim}


The distribution's codename can be found using the command lsb\_release -c. 
e.g.
\begin{tinyverbatim}
lsb_release -c
Codename: wheezy
\end{tinyverbatim}

After the repository configuration the installation is completed with two simple commands.

\begin{tinyverbatim}
apt-get update
apt-get install postgreql-9.4 postgreql-contrib-9.4 postgreql-client-9.4
\end{tinyverbatim}

This method creates automatically a new database cluster in the home directory /var/lib/postgresql.

