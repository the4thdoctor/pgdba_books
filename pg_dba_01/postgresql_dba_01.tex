\documentclass[oneside]{book}
\usepackage[utf8]{inputenc}
\usepackage{makeidx}
\usepackage{graphicx}
\setcounter{tocdepth}{2}
\usepackage{hyperref}
\hypersetup{pdfborder={0 0 0},colorlinks=false}
\usepackage{verbatim}
\newenvironment{smallverbatim}{\endgraf\small\verbatim}{\endverbatim}
\newenvironment{tinyverbatim}{\endgraf\scriptsize\verbatim}{\endverbatim}
\pagestyle{plain}
\usepackage{float}
\usepackage{pdfpages}
\usepackage{listings}
\usepackage[paperwidth=21.59cm,paperheight=27.94cm]{geometry}


\author{Federico Campoli}
\title{PostgreSQL Database Administration \\ Volume 1 \\ Basic concepts}


\makeindex

\lstdefinestyle{pgsql}{
  belowcaptionskip=1\baselineskip,
  breaklines=true,
  frame=l,
  language=SQL,
  showstringspaces=false,
  basicstyle=\footnotesize\ttfamily,
  keywordstyle=\bfseries\color{green!40!black},
  commentstyle=\itshape\color{purple!40!black},
  identifierstyle=\color{blue},
  stringstyle=\color{orange},
  morekeywords={VACUUM, FULL, ANALYZE, TABLESPACE,SET}
}


\begin{document}
\includepdf{covers/cover_good.pdf}
\maketitle

\newpage{}



\tableofcontents{}

\chapter*{Preface}
When I first came up with the idea to write a PostgreSQL DBA book, my intention was go 
through a commercial publisher.\newline

Shortly I changed my mind as I became aware the uniqueness of a PostgreSQL DBA oriented book.
I took the decision to keep this book, and all the subsequent I'll write, free. This 
hopefully will spread the PostgreSQL knowledge becoming a reference to help the 
PostgreSQL professionals.\newline

Before you start reading this book I'm begging your pardon in advance for my bad English.\newline

As I did not study English and unfortunately my mothertongue is not English. The following text 
will very likely be full of typos (I'm affected by some dyslexia as well) and bad grammar.\newline

Anyway, my promise  is this book and all the subsequent in the series, will be free as freedom 
and as beer.\newline
However, if you want to help me to cleanup the text you're very welcome to fork the github 
repository where I'm storing the latex sources \href{https://github.com/the4thdoctor/pgdba\_books}{
https://github.com/the4thdoctor/pgdba\_books}.\newline




\section*{Intended audience}
Database administrators, System administrators, Developers

\section*{Book structure}
This book assumes the reader knows how to perform basic user operations such as
connecting to the database and creating tables.

The book covers the basic aspects of database administration from installation
to cluster management.

A couple of chapters are dedicated to the logical and physical structure in
order to show both sides of coin.  Maintenance and disaster recovery
backup/restore completes the ``other side of the monitor's`` picture.
The final chapter is for developers who want to avoid common pitfalls when
using PostgreSQL.

\section*{Version and platform}
This book is based on PostgreSQL version 9.3 running on Debian GNU Linux 7.
References to older versions or different platforms are explicitly specified.

\section*{Thanks}
A big thank you to Craig Barnes for the priceless work on the book review.\newline
The beautiful cover has been made by \href{http://www.bonland.eu/}{
Chiaretta e Bon }.\newline


\chapter{PostgreSQL at a glance}
PostgreSQL is a first class product with enterprise class features.
This chapter is nothing but a general review on the product with a short section dedicated to the 
database's history.

\section{Long time ago in a galaxy far far away...}

Following the works of the Berkeley's Professor  Michael Stonebraker, in 1996
Marc G. Fournier\index{Marc G. Fournier} asked if there was volunteers interested
in revamping the Postgres 95 project.\newline
\begin{smallverbatim}

Date: Mon, 08 Jul 1996 22:12:19-0400 (EDT) 
From: ”Marc G. Fournier” <scrappy@ki.net>
Subject: [PG95]: Developers interested in improving PG95?
To: Postgres 95 Users <postgres95@oozoo.vnet.net>
Hi... A while back, there was talk of a TODO list and development moving forward on Postgres95 ...
at which point in time I volunteered to put up a cvs archive and sup server so that making updates 
(and getting at the ”newest source code”) was easier to do...
... Just got the sup server up and running, and for those that are familiar with sup, the following 
should work (ie. I can access the sup server from my machines using this): 

..........................
\end{smallverbatim}


At this email replied  Bruce Momjian\index{Bruce Momjian},Thomas Lockhart\index{Thomas Lockhart}, 
and Vadim Mikheev\index{Vadim Mikheev}, the very first PostgreSQL Global Development Team.\newline 

Today, after almost 20 years and millions of rows of code, PostgreSQL is a robust and reliable 
relational database. The most advanced open source database. The slogan speaks truth indeed.

\section{Features}
Each time a new major release is released it adds new features to the already 
rich feature's set. What follows is a small excerpt of the latest PostgreSQL's version 
capabilities. 

\subsection{ACID compliant}
The word ACID is an acronym for Atomicity, Consistency, Isolation and Durability. An 
ACID compliant database ensures those ules are enforced at any time. \newline
\begin{itemize}

\item The atomiticy is enforced when a transaction is ``all or nothing''. For example if a 
transaction inserts a group of new rows. If just one row violates the primary key then the entire 
transaction must be rolled back leaving the table as nothing happened.

\item The consistency ensures the database is constantly in a valid state. The database steps from 
a valid state to another valid state without exceptions.

 \item The isolation is enforced when the database status can be reached like all the concurrent 
transactions were run in serial. 

\item The durability ensures the committed transactions are saved on durable storage. In the event 
of the database crash the database must recover to last valid state.

\end{itemize}

\subsection{MVCC}
PostgreSQL ensures atomiticy consistency and isolation via the MVCC. The acronym stands 
for Multi Version Concurrency Control. The mechanism is incredibly efficient, it offers
great level of concurrency keeping the transaction's snapshots isolated and consistent. There is 
one single disadvantage in the implementation. We'll see in detail in \ref{sec:MVCC} how MVCC works 
and the reason why there's no such thing like an update in PostgreSQL.

\subsection{Write ahead logging}
The durability is implemented in PostgreSQL using the write ahead log.
In short, when a data page is updated in the volatile memory the change is saved immediately on 
a durable location, the write ahead log\index{wal}\index{write ahead log}. The page is written 
on the corresponding data file later. In the event of the database crash the write ahead log is 
scanned and all the not consistent pages are replayed on the data files.
Each segment size is usually 16 MB and their presence is automatically managed by PostgreSQL. 
The write happens in sequence from the segment's top to the bottom. When this is full PostgreSQL 
switches to a new one. When this happens there is a log switch. \index{log switch}

\subsection{Point in time recovery}
\index{pitr}\index{point in time recovery}\index{log shipping}When PostgreSQL switches to a new 
WAL this could be a new segment or a recycled one. If the old WAL is archived in a safe
location it's possible to get a copy of the physical data files meanwhile the database is running. 
The hot copy, alongside with the archived WAL segments have all the informations necessary and 
sufficient to recover the database's consistent state. The recovery by default terminates when all 
the archived data files are replayed. Anyway it's possible to stop the recover at a given point in 
time. 

\subsection{Standby server and high availability}
\index{standby server}\index{high availability}The standby server is a database configured to 
stay in continuous recovery. This way a new archived WAL file is replayed as soon as it becomes 
available. This feature was first in introduced with  PostgreSQL 8.4 as warm standby\index{warm 
standby}. From the version 9.0 PostgreSQL can be configured also in hot standby\index{hot standby}  
which allows the connections for read only queries.

\subsection{Streaming replication}
The WAL archiving doesn't work in real time. The segment is shipped only after a log switch and 
in a low activity server this can leave the standby behind the master for a while. It's  possible 
to limit the problem using the archive\_timeout parameter which forces a log swith after the given 
number of seconds. However, using the streaming replication\index{streaming replication} a standby 
server can get the wal blocks over a database connection in almost real time. This feature allows 
the physical blocks to be transmitted over a conventional database connection. The version 9.4 introduces 
the replication slots which makes simple to manage multiple slaves attached to the master.

\subsection{ALTER SYSTEM}
This long waited feature permits changing the cluster's configuration using the 


\subsection{Procedural languages}
PostgreSQL have many procedural languages. Alongside with the pl/pgsql it's possible to write the 
procedure in many other popular languages like pl/perl and pl/python. From the version 9.1 is also 
supported the anonymous function's code block with the DO keyword.

\subsection{Partitioning}
Despite the partitioning\index{partitioning}\index{constraint exclusion} implementation in
PostgreSQL is still very basic it's not complicated to build an efficient partitioned structure 
using 
the table inheritance.\newline

Unfortunately because the physical storage is distinct for each partition, is not possible to 
have a global primary key for the partitioned structure. The foreign keys can be emulated in some 
way using the triggers.

\subsection{Cost based optimiser}
The cost based optimiser, or CBO,\index{cost based optimizer}\index{CBO} is one of the PostgreSQL's 
point of strength The execution plan is dynamically determined from the data distribution and from 
the query parameters. PostgreSQL also supports the genetic query optimizer GEQO.


\subsection{Multi platform support}
PostgreSQL\index{platform} supports almost any unix flavour, and from version 8.0 runs natively on 
Windows.

\subsection{Tablespaces}
The tablespace support permits a fine grained distribution of the data files across
filesystems. In \ref{sub:TBS-LOGICAL} and \ref{sub:TBS-PHYSICAL} we'll see how to take advantage of 
this powerful feature.

\subsection{Triggers} 
The triggers are well supported on tables and views. A basic implementation of the events 
triggers is also present. The triggers can emulate completely the updatable views feature. 

\subsection{Views}
The read only views are well consodlidated in PostgreSQL.
The version 9.3 introduced the basic support for the materialised and updatable views.
For the materialised views there is no incremental refresh. The complex views, like views 
joining two or more tables, are not updatable. 

\subsection{Constraint enforcement}
PostgreSQL supports primary keys and unique keys to enforce table's data. The referential integrity 
is guaranteed with the foreign keys. We'll take a look to the data integrity in \ref{cha:DATAINT}

\subsection{Extension system}
PostgreSQL from the version 9.1 implements a very efficient extension system. The command CREATE 
EXTENSION makes the new features installation easy.

\subsection{Federated}
From PostgreSQL 9.1 is possible to have foreign tables pointing to external data sources. 
PostgreSQL 9.3 introduced also the foreign table's write the PostgreSQL's foreign data 
wrapper.
\chapter{Database installation}
\label{cha:DB_INSTALL}
This chapter will cover the install procedure, on Debian Gnu linux compiling from source and using the packaged install from the pgdg archive.
\index{pgdg}

\section{Install from source}
Installing from source, using the default configuration settings requires the root access as the default install location is in /usr/local/. 
To simplify things I've created a procedure with minimal need for root access.
This of course is still required but only for the os user creation and to install the dependencies. 

Before starting with the postgresql part, ask your sysadmin, or do it by yourself, to do the following

\begin{itemize}

 \item Create a postgres group and a postgres user
 \item Add the postgres user to the postgres group
 \item Install the packages
 \item build-essential
 \item libreadline6-dev
 \item zlib1g-dev
 
\end{itemize}


When everything is in place login as postgres user and download the source's tarball.

\begin{smallverbatim}
mkdir download
cd download
wget http://ftp.postgresql.org/pub/source/v9.3.4/postgresql-9.3.4.tar.bz2
\end{smallverbatim}

Then extract the tarball with

\begin{smallverbatim}
tar xfj postgresql-9.3.4.tar.bz2
cd postgresql-9.3.4
\end{smallverbatim}

Using the configuration’s script option --prefix is possible to change 
the install directory to a custom location.
Assuming the postgres user have his home directory in /home/postgres, we'll put 
the install target into the bin directory organised per
mayor version. In this way  is possible to have multiple versions on the same 
box without hassle.

\begin{smallverbatim}
mkdir -p /home/postgres/bin/9.3
./configure --prefix=/home/postgres/bin/9.3
\end{smallverbatim}
The script will check all the dependencies and will generate the makefiles. 
Any error at configure time is usually displayed clearly.

When everything looks fine you can start the build process with the \textit{make} command.

The time required to compile depends from the box speed. On a laptop usually this doesn't require 
more than 30 minutes. After the build is complete it's a good idea to run the regression tests 
before completing the installation.

\begin{smallverbatim}
make check 
\end{smallverbatim}
 

All the output is written in the directory src/test/regress/results. 

If everything looks fine the installation can be completed with

\begin{smallverbatim}
make install
\end{smallverbatim}


When the install is complete, into the /home/postgres/bin/9.3 directory will appear with 4 new
subfolders \textit{bin} \textit{include} \textit{lib} and \textit{share}.

\begin{itemize}
 \item \textbf{bin} is where the database binaries are stored
 \item \textbf{include} contains the server's header files
 \item \textbf{lib} is the shared objects location
 \item \textbf{share} is where the example files and the extension config are stored
\end{itemize}



\section{Packaged install}
\label{sec:DEBIAN_INSTALL}

The PostgreSQL Global Group mantains an apt repository to simplify the install on the GNU/Linux based on debian.

The supported Linux versions are listed on the wiki page http://wiki.postgresql.org/wiki/Apt and at moment are

\begin{itemize}
 \item Debian 6.0 (squeeze)
 \item Debian 7.0 (wheezy)
 \item Debian unstable (sid) 
 \item Ubuntu 10.04 (lucid)
 \item Ubuntu 12.04 (precise)
 \item Ubuntu 13.10 (saucy)
 \item Ubuntu 14.04 (trusty) 
\end{itemize}
The packages are available for amd64 and i386.

The available database versions are
\begin{itemize}
 \item PostgreSQL 8.4 
 \item PostgreSQL 9.0 
 \item PostgreSQL 9.1 
 \item PostgreSQL 9.2 
 \item PostgreSQL 9.3
\end{itemize}

Before starting you shall import the GPG key to validate the packages.

In a root shell simply run
\begin{smallverbatim}
wget --quiet -O - https://www.postgresql.org/media/keys/ACCC4CF8.asc | sudo apt-key add -
\end{smallverbatim}
Then add the file pgdg.list into the directory /etc/apt/sources.d/ with the following contents

\begin{smallverbatim}
deb http://apt.postgresql.org/pub/repos/apt/ {codename}-pgdg main
\end{smallverbatim}

Change the codename value accordingly with your distribuition. (e.g. wheezy) then you can run the 
installation in two simple steps.

\begin{smallverbatim}
apt-get update
apt-get install postgreql-9.3 postgreql-contrib-9.3 postgreql-client-9.3 
\end{smallverbatim}
The debian's packaged installation as post installation task, if not yet present, will create a
new running database cluster in the home directory /var/lib/postgresql.


\chapter{Install structure}\label{cha:INSTALLSTRUCT}
Depending on the installation method, the install structure is set up in a single directory or 
in multiple folders.\newline

The install from source creates four subfolders in the target directory: \textit{bin},
\textit{include}, \textit{lib} and \textit{share}.

\begin{itemize}
 \item \textbf{bin} contains the PostgreSQL binaries
 \item \textbf{include} contains the server's header files
 \item \textbf{lib} contains the shared libraries
 \item \textbf{share} contains the example files and the extension configurations
\end{itemize}


The packaged install puts the binaries and the libraries in the folder /usr/lib/postgresql 
organised by major version. For example the 9.3 install will put the binaries into 
/usr/lib/postgresql/9.3/bin and the libraries in /usr/lib/postgresql/9.3/lib. The extensions and 
contributed modules are installed into the folder /usr/share/postgresql with the same structure. The
Debian specific utilities and the symbolic link to the psql binary (which is at 
/usr/lib/share/postgresql-common/pg\_wrapper) are stored in the directory /usr/bin/. This file is a perl script which 
calls the PostgreSQL client reading the version the cluster and the default database from the file 
~/.postgresqlrc or /etc/postgresql-common/user\_clusters.\newline


\section{The core binaries}
The PostgreSQL binaries can be split in two groups, the core and the wrappers alongside with the 
contributed modules. Let's start then with the former group.

\subsection{postgres}\index{postgres}
\label{sec:POSTGRES}
This is the PostgreSQL's main process. The program can be started directly or using the pg\_ctl 
utility. The second method is to be preferred as it offers a simpler way to control the postgres 
process. Direct execution is the unavoidable choice when the database won't start for an old XID 
near to the wraparound failure\index{XID wraparound failure}. 
In this case the cluster can only start in single user mode to perform a cluster wide vacuum. For 
historical reasons there's also a symbolic link named postmaster pointing to the postgres 
executable.

\subsection{pg\_ctl}\index{pg\_ctl}
\label{sub:PGCTL}
This utility is the simplest way for managing a PostgreSQL instance. The program reads the postgres 
pid from the cluster's data area and sends the os signals to start, stop or reload the
process. It's also possible to send kill signals to the running instance. 
pg\_ctl provides the following actions:

\begin{itemize}
 \item \textbf{init[db]} initialises a directory as PostgreSQL data area
 \item \textbf{start} starts a PostgreSQL instance
 \item \textbf{stop} shutdowns a PostgreSQL instance
 \item \textbf{reload} reloads the configuration's files
 \item \textbf{status} checks the PostgreSQL instance running status
 \item \textbf{promote} promotes a standby server 
 \item \textbf{kill} sends a custom signal to the running instance
\end{itemize}

In \ref{cha:MANAGING} we'll se how to manage the cluster.

\subsection{initdb}\index{initdb}
Is the binary which initialises the PostgreSQL data area. The directory to initialise must 
be empty. Various options can be specified on the command line, like the character encoding or the 
collation order. 

\subsection{psql}\index{psql}
% TODO: Review statement 'The client it looks very essential'
This is the PostgreSQL command line client. The client it looks very essential, however is one of 
the most flexible tools available to interact with the server and the only choice when working on 
the command line.

\subsection{pg\_dump}\index{pg\_dump}
\label{sub:PGDUMP}
This is the binary dedicated to backup. Can produce consistent backups in various formats. The 
usage is described shown in \ref{cha:BACKUP}.

\subsection{pg\_restore}\index{pg\_restore}
This program is used to restore a database reading a binary dump like the custom or directory 
format. It's able to run the restore in multiple jobs in order to speed up the process. The usage 
is described in \ref{cha:RESTORE}

\subsection{pg\_controldata}\index{pg\_controldata}\label{sub:PGCONTROLDATA}
This program can query the cluster's control file where PostgreSQL stores critical information about 
the cluster activity and reliability. 

\subsection{pg\_resetxlog}\index{pg\_resetxlog}
If a WAL file becomes corrupted the cluster cannot perform a crash recovery. This lead to an
unstartable cluster in case of system crash. In this catastrophic scenario there's still a 
way to start the cluster. Using pg\_resetxlog the cluster is cleared of any WAL file, the  
control file is initialised from scratch and the transaction's count is restarted.\newline

The \textit{tabula rasa} comes with a cost indeed. The cluster loses any reference between the 
transaction progression and the data files. The physical integrity is lost and any attempt to run 
queries which write data will result in corruption.\newline 

The PostgreSQL's documentation is absolutely clear on this point.

\begin{verbatim}

After running pg_resetxlog the database must start without user access, 
the entire content must be dumped, the data directory must be dropped and recreated 
from scratch using initdb and then the dump file can be restored using psql or pg_restore
\end{verbatim}

\section{Wrappers and contributed modules}
The second group of binaries is composed of the contributions and the wrappers. The 
contributed modules add functions otherwise not available. The wrappers add command line 
functions already present as SQL statements. Someone will notice the lack of HA specific binaries 
like pg\_receivexlog and pg\_archivecleanup. They have been purposely skipped because they are beyond the 
scope of this book.

\subsection{The create and drop utilities}
The binaries with the prefix create and drop like, createdb createlang createuser and dropdb, 
droplang, dropuser, are wrappers for the corresponding SQL functions. Each program performs the 
creation and the drop action on the corresponding named object. For example createdb adds a 
database to the cluster and dropdb will drop the specified database. 

\subsection{clusterdb}\index{clusterdb}
This program performs a database wide cluster on the tables with clustered indices. 
The binary can run on a single table specified on the command line. In \ref{sec:VACFULL} we'll 
take a look to CLUSTER and VACUUM FULL.

\subsection{reindexdb}\index{reindexdb}
The command does a database wide reindex. It's possible to run the command just on a table or index 
passing the relation's name on the command line. In \ref{sec:REINDEX} we'll take a good look at 
the index management.

\subsection{vacuumdb}\index{vacuumdb}
This binary is a wrapper for the VACUUM \index{VACUUM} SQL command. This is the most important 
maintenance task and shouldn't be ignored. The program performs a database wide VACUUM if executed 
without a target relation. Alongside with a common vacuum it's possible to have the usage 
statistics updated on the same time.

\subsection{vacuumlo}\index{vacuumlo}
This binary will remove the orphaned large objects from the pg\_largeobject system table. The 
pg\_largeobject is used to store the binary objects bigger than the limit of 1GB imposed by the 
bytea data type. The limit for a large object it is 2 GB since the version 9.2. From the version 
9.3 the limit was increased to 4 TB. 

\section{Debian's specific utilities}
Finally let's take a look at the Debian specific utilities. They are a collection of perl scripts 
used to simplify the cluster's management. They are installed in /usr/bin and mostly consist of symbolic 
links to the actual executable. We already mentioned one of them in the chapter's introduction, the 
psql pointing to the pg\_wrapper PERL script.

\subsection{pg\_createcluster}\index{pg\_createcluster}
This script adds a new PostgreSQL cluster with the given major version, if installed, and the 
given name. The script puts all the configuration in /etc/postgresql. Each major version has a 
dedicated directory within which is a group of directories containing the cluster's specific 
configuration files. If not specified the data directory is created in the folder 
/var/lib/postgresql. It's possible to specify the options for initd.

\subsection{pg\_dropcluster}\index{pg\_dropcluster}
The program will delete a PostgreSQL cluster created previously with pg\_createcluster. The 
program will not drop a running cluster. If the dropped cluster has any tablespace those must be 
manually removed after the drop as the program doesn't follow the symbolic links.

\subsection{pg\_lscluster}\index{pg\_lscluster}
Lists the clusters created with pg\_createcluster.

\subsection{pg\_ctlcluster}\index{pg\_ctlcluster}
\label{sub:PGCTLDEB}
% TODO: Understand this better to rewrite
The program manages the cluster in a similar way pg\_ctl does. 
Before version 9.2 this wrapper had dangerous behaviour for the shutdown. The script did not 
offered a flexible way to provide the shutdown mode. More informations about the shutdown 
sequence are in \ref{sec:SHUTDOWN_SEQ}. 
When run without any arguments pg\_ctlcluster performs a smart shutdown mode.
The --force option tells the script to try a \textit{fast} shutdown mode. Unfortunately if the 
database doesn't shutdown in a \textit{reasonable time} the script performs an \textit{immediate} 
shutdown. After another short wait, if the the instance is still up the script sends a 
\textit{kill -9} to the postgres process. Because these kind of actions can result in data loss  
they should be done manually by the DBA. It's better to avoid using pg\_ctlcluster for the shutdown.

\chapter{Managing the cluster}
\label{cha:MANAGING}
A PostgreSQL cluster is made of two components. A physical location initialised as data area 
and the postgres process attached to a shared memory segment, the shared buffer. The debian's 
package's installation, automatically set up a fully functional PostgreSQL cluster in the directory 
/var/lib/postgresql. This is good because it's possible to explore the product immediately.
However, 
it's not uncommon to find clusters used in production with the minimal default configuration's 
values, just because the binary installation does not make it clear what happens \textit{under the 
bonnet}.

This chapter will explain how a PostgreSQL cluster works and how critical is its 
management. 

\section{Initialising the data directory}

The data area is initialised by initdb\index{initdb}. The program requires an empty directory to 
write into to successful complete. Where the initdb binary is located depends from the installation 
method. We already discussed of this in \ref{cha:INSTALLSTRUCT} and \ref{cha:DB_INSTALL}. 

The accepted parameters for customising cluster's data area are various. Anyway, running 
initdb without parameters will make the program to use the value stored into the environment 
variable PGDATA. If the variable is unset the program will exit without any further action.\newline

For example, using the initdb shipped with the debian archive requires the following commands.

\begin{verbatim}
postgres@tardis:~/$ mkdir tempdata
postgres@tardis:~/$ cd tempdata
postgres@tardis:~/tempdata$ export PGDATA=`pwd`
postgres@tardis:~/tempdata$ /usr/lib/postgresql/9.3/bin/initdb 
The files belonging to this database system will be owned by user "postgres".
This user must also own the server process.

The database cluster will be initialized with locale "en_GB.UTF-8".
The default database encoding has accordingly been set to "UTF8".
The default text search configuration will be set to "english".

Data page checksums are disabled.

fixing permissions on existing directory /var/lib/postgresql/tempdata ... ok
creating subdirectories ... ok
selecting default max_connections ... 100
selecting default shared_buffers ... 128MB
creating configuration files ... ok
creating template1 database in /var/lib/postgresql/tempdata/base/1 ... ok
initializing pg_authid ... ok
initializing dependencies ... ok
creating system views ... ok
loading system objects' descriptions ... ok
creating collations ... ok
creating conversions ... ok
creating dictionaries ... ok
setting privileges on built-in objects ... ok
creating information schema ... ok
loading PL/pgSQL server-side language ... ok
vacuuming database template1 ... ok
copying template1 to template0 ... ok
copying template1 to postgres ... ok
syncing data to disk ... ok

WARNING: enabling "trust" authentication for local connections
You can change this by editing pg_hba.conf or using the option -A, or
--auth-local and --auth-host, the next time you run initdb.

Success. You can now start the database server using:

    /usr/lib/postgresql/9.3/bin/postgres -D /var/lib/postgresql/tempdata
or
    /usr/lib/postgresql/9.3/bin/pg_ctl -D /var/lib/postgresql/tempdata -l 
logfile start

\end{verbatim}

PostgreSQL 9.3 introduces\index{checksums, data page}\index{data page checksums} the data page 
checksums used for detecting the data page corruption. This great feature can be enabled only when 
initialising the data area with initdb and is cluster wide. The extra overhead caused by the 
checksums is something to consider because the only way to disable the data checksums is a dump 
and reload on a fresh data area.\newline

After initialising the data directory initdb emits the message with the commands to start the 
database cluster. The first form is useful for debugging and development purposes because it starts 
the database directly from the command line with the output displayed on the terminal. 

\begin{verbatim}
postgres@tardis:~/tempdata$ /usr/lib/postgresql/9.3/bin/postgres -D 
/var/lib/postgresql/tempdata
LOG:  database system was shut down at 2014-03-23 18:52:07 UTC
LOG:  database system is ready to accept connections
LOG:  autovacuum launcher started

\end{verbatim}

Pressing CTRL+C stops the cluster with a fast shutdown.\newline

Another reason for running postgres directly is when it needs to be started in single user mode. 
The --single option is a lifesaver if the cluster refuses to start because one or more 
databases are near the XID wraparound failure. 
\begin{verbatim}

postgres@tardis:~/tempdata$ /usr/lib/postgresql/9.3/bin/postgres --single -D /home/postgres/tempdata

PostgreSQL stand-alone backend 9.3.5
backend> 

\end{verbatim}

The database interface in single user mode and does not have all the sophisticated features 
like the client psql. Anywat with a little knowledge of SQL it's possible to find the database(s) 
causing the shutdown and fix it.
\index{postgres, single user mode}\index{XID wraparound failure, fix}

\begin{verbatim}
backend> SELECT datname,age(datfrozenxid) FROM pg_database ORDER BY 2 DESC;
         1: datname     (typeid = 19, len = 64, typmod = -1, byval = f)
         2: age (typeid = 23, len = 4, typmod = -1, byval = t)
        ----
         1: datname = "template1"       (typeid = 19, len = 64, typmod = -1, byval = f)
         2: age = "2146435072"  (typeid = 23, len = 4, typmod = -1, byval = t)
        ----
         1: datname = "template0"       (typeid = 19, len = 64, typmod = -1, byval = f)
         2: age = "10"  (typeid = 23, len = 4, typmod = -1, byval = t)
        ----
         1: datname = "postgres"        (typeid = 19, len = 64, typmod = -1, byval = f)
         2: age = "10"  (typeid = 23, len = 4, typmod = -1, byval = t)
        ----

\end{verbatim}

The age function shows how old is the last XID not yet frozen. In our example the template1
database have an age of 2146435072, one million transactions to the wraparound. We can then exit 
the backend with CTRL+D and restart it again in the in single user mode specifying the database 
name. A VACUUM will get rid of the problematic xid.

\begin{verbatim}
postgres@tardis:~/tempdata$ /usr/lib/postgresql/9.3/bin/postgres --single \
-D /home/postgres/tempdata template1

backend> VACUUM;
\end{verbatim}

This procedure must be repeated for any database with very old XID.\newline

Starting the cluster with pg\_ctl usage is very simple. This program also accepts the data area as 
parameter or using the environment variable PGDATA. It's also required to provide the command to 
execute. The start command for example is used to start the cluster in multi user mode.

\begin{verbatim}

postgres@tardis:~/tempdata$ /usr/lib/postgresql/9.3/bin/pg_ctl -D 
/var/lib/postgresql/tempdata -l logfile start
server starting

postgres@tardis:~/tempdata$ tail logfile 
LOG:  database system was shut down at 2014-03-23 19:01:19 UTC
LOG:  database system is ready to accept connections
LOG:  autovacuum launcher started

\end{verbatim}
Omitting the logfile with the -l will display the alerts and warnings on the terminal.

The command stop will end the cluster's activity.

\begin{verbatim}
postgres@tardis:~$ /usr/lib/postgresql/9.3/bin/pg_ctl -D 
/var/lib/postgresql/tempdata -l logfile stop
waiting for server to shut down.... done
server stopped
\end{verbatim}

\section{The startup sequence}
\label{sec:STARTUP}

When the postgres process starts it allocates the shared memory segment called shared buffer. The
size of this segment is specified with the GUC parameter shared\_buffers.The version 9.3, on the
unix systems, changed the memory allocation method to mmap(). This eliminated any need to adjust
the kernel's parameters.

For the versions up to the 9.2, if the requested memory is bigger than the kernel's maximum
allowed size, the startup sequence will abort with an error like this.

\begin{verbatim}
FATAL: could not create shared memory segment: Cannot allocate memory

DETAIL: Failed system call was shmget(key=X, size=XXXXXX, XXXXX).

HINT: This error usually means that PostgreSQL's request for a shared memory
segment exceeded available memory or swap space, or exceeded your kernel's
SHMALL parameter. You can either reduce the request size or reconfigure the
kernel with larger SHMALL. To reduce the request size (currently XXXXX bytes),
reduce PostgreSQL's shared memory usage, perhaps by reducing shared_buffers or
max_connections.
\end{verbatim}

\index{kernel resources}
The kernel's parameters governing this limit is SHMMAX. It sets the maximum allowed size of a shared
memory segment. The value is measured in bytes and must be big enogh to contain the
requested shared\_buffers. Another parameter which needs adjustment is SHMALL. This value sets the
amount of total shared memory available. On linux is usually measured in pages. Unless the
kernel is configured to allow the huge pages the page size is 4096 byes. The value should be the
same as SHMMAX. \newline

If we want to set a pre 9.3 with a shared buffer to 1 GB the SHMMAX should be at least 1073741824.
The value 1258291200 (1200 MB) is a reasonable setting ang gives us some extra headroom. The
corresponding SHMALL is 307200. The value SHMMNI is the minimum value of the shared memory, is safe
to set to 4096, just one memory page. 

The settings can be changed on the fly, simply echoing in the corresponding proc entries or setting
the values persistently into the file /etc/sysctl.conf.

Here's the file from the previous example.
\begin{verbatim}
kernel.shmmax = 1258291200
kernel.shmall = 307200
kernel.shmmni = 4096
kernel.sem = 250 32000 100 128
fs.file-max = 658576
\end{verbatim}

To apply the changes login as root and run \textit{sysctl -p}.\newline


When the memory is allocated the postmaster reads the pg\_control
file to check if the instance requires recovery. The pg\_control file is used to store the locations
to the last checkpoint and the last known status for the instance.\newline

If  the instance is in dirty state, because a crash or an unclean shutdown, the startup
process reads the last checkpoint location and replays the blocks from the corresponding WAL
segment in the pg\_xlog directory. Any corruption in the wal files during the recovery or the
pg\_control file results in a not startable instance.\newline

When the recovery is complete or if the cluster's state is clean the postgres process completes the
startup and sets the cluster in production state. 


\section{The shutdown sequence} 
\label{sec:SHUTDOWN_SEQ}
\index{shutdown sequence}

The PostgreSQL process enters the shutdown status when a specific OS signal is received. The signal
can be sent via the os kill or using the program pg\_ctl. \newline

As seen in \ref{sub:PGCTL} pg\_ctl accepts the -m switch when the command is stop. The -m switch
is used to specify the shutdown mode and if is omitted it defaults to smart which corresponds to
the SIGTERM signal. With the smart shuthdown the cluster stops accepting new connections and
waits for all backends to quit. \newline

When the shutdown mode is set to fast pg\_ctl sends the SIGQUIT signal to the postgres main process.
Same as for the smart shutdown the cluster does not accepts new connections terminates the existing
backends. Any open transaction is rolled back as well. \newline

When the smart and the fast shutdown are complete they leave the cluster in clean state. This is
true because when the postgres process initiate the final part of the shutdown it starts a
last checkpoint which consolidates any dirty block on the disk. Before quitting the postgres
process saves the latest checkpoint's location to the pg\_control file and marks the
cluster as clean.\newline

The checkpoint can slow down the entire shutdown sequence. In particular if the shared\_buffer is
big and contains many dirty blocks, the checkpoint can run for a very long time. Also if at
the shutdown time, another checkpoint is running the postgres process will wait for this
checkpoint to complete before starting the final checkpoint.\newline

Enabling the log checkpoints in the configuration gives us some visibility on what the cluster is
actually doing. The GUC parameter governing the setting is log\_checkpoints.\newline


If the cluster doesn't stop, there is a shutdown mode which leaves the cluster in dirty state.
The immiediate shutdown. The equivalent signal is the SIGQUIT and it causes the main process
alongside with the backends to quit immediately without the checkpoint.\newline

The subsequent start will require a crash recovery. The recovery is usually harmless with one
important exception. If the cluster contains unlogged tables those relations are recreated from
scratch when the recovery happens and all the data in those table is lost.

A final word about the SIGKILL signal, the dreaded kill -9. It can happen the cluster refuse to
stop even using the immediate mode. In this case, as last resort the SIGKILL. Because this signal
cannot be trapped in any way, the resources like the shared memory and the inter process semaphores
will stay in place after the kill. This will very likely affect the start of a fresh instance.
Please refer to your sysadmin to find out the best way to cleanup the memory after the
SIGKILL.

\section{The processes}
\label{sec:PROCESSES}
Alongside with postgres process there are a number of accessory processes. With a running 9.3
cluster ps shows at least six postgres processes. 

\subsection{postgres: checkpointer process}
As the name suggests this process take care of the cluster's checkpoint\index{checkpoint} activity.
A checkpoint is an important event in the cluster's life. When it starts all the dirty pages in
memory are written to the data files. The checkpoint frequency is regulated by the time and the
number of cluster's WAL switches.The GUC parameters governing this metrics are respectively
checkpoint\_timeout\index{checkpoint\_timeout} and checkpoint\_segments\index{checkpoint\_segments}.
There is a third parameter, the checkpoint\_completion\_target\index{checkpoint\_completion\_target}
which sets the percentage of the checkpoint\_timeout. The cluster uses this value to spread the
checkpoint over this time in order to avoid a big disk IO spike.

\subsection{postgres: writer process}
The background writer scans the shared buffer searching for dirty pages which writes on the data
files. The process is designed to have a minimal impact on the database activity. It's possible to
tune the length of a run and the delay between the writer's runs using the GUC parameters
bgwriter\_lru\_maxpages\index{bgwriter\_lru\_maxpages} and bgwriter\_delay\index{bgwriter\_delay}.
They are respectively the number of dirty buffers written before the writer's sleep and the time
between two runs.

\subsection{postgres: wal writer process}
This background process has been introduced with the 9.3 in order to make the WAL writes a more
efficient. The process works in rounds and writes down the wal buffers to the  wal files. The GUC
parameter wal\_writer\_delay\index{wal\_writer\_delay} sets the milliseconds to sleep between the
rounds. 

\subsection{postgres: autovacuum launcher process}
This process is present if the autovacuum\index{autovacuum} is enabled. It's purpose is to launch
the autovacuum backends when needed. 

\subsection{postgres: stats collector process}
The process gathers the database's usage statistics and stores the information to the location
indicated by the GUC stats\_temp\_directory. This is by default pg\_stat\_temp, a relative path to
the data area. 

\subsection{postgres: postgres postgres [local] idle}
This is a database backend. There is one backend for each established connection. The values after
the colon show useful information. In particular between the square brackets there is the query
the backend is executing. 

\section{The memory}
\label{sec:MEMORY}
The PostgreSQL's memory structure is not complex like other databases.
In this section we'll take a to the various parts. 

\subsection{The shared buffer}
\index{shared buffer}
The shared buffer, as the name suggests is the segment of shared memory used by 
PostgreSQL to manage the data pages. 

Its size is set using the GUC\footnote{GUC, Grand Unified Configuration, this 
acronym refers to the parameters used to configure the instance} parameter 
shared\_buffers and is allocated 
during the startup process.Any change requires the instance restart.

The segment is formatted in blocks with the same size of the data file's 
blocks, usually 8192 bytes. Each backend connected to the cluster is attached 
to this segment. Because usually its size is a fraction of the cluster's size, 
a simple but very efficient mechanism keeps in memory the blocks using a 
combination of LRU and MRU.

Since the the version 8.3 is present a protection mechanism to avoid the 
massive block eviction when intensive IO operations, like vacuum or big 
sequential reads, happens.

Each database operation, read or write, is performed moving the blocks via the 
shared buffer. This ensure an effective caching process and the memory routines 
guarantee the consistent read and write at any time.

PostgreSQL, in order to protect the shared buffer from potential corruption, if 
any unclean disconnection happens, resets by default all the connections. 

This behaviour can be disabled in the configuration file but exposes the shared 
buffer to data corruption if the unclean disconnections are not correctly 
managed.



\subsection{The work memory}\index{work memory}
\label{sub:WORKMEM}
This memory segment is allocated per user and its default value is set using 
the GUC parameter work\_mem. The value can be altered for the session on the 
fly. When changed in the global configuration file becomes effective to the 
next transaction after the instance reload. 

This segment is used mainly for expensive operations like the sort or the 
hash.

If the operation's memory usage exceeds the work\_mem value then the PostgreSQL 
switches to a disk sort/hash. 

Increasing the work\_mem value results generally in better performances for 
sort/hash operations. 

Because is a per user memory segment, the potential amount of memory 
required in a running instance is max\_connections * work\_mem. It's very 
important to set this value to a reasonable size in order to avoid any risk of 
out of memory error or unwanted swap.

In complex queries is likely to have many sort or hash operations in parallel 
and each one consumes the amount of work\_mem for the 
session.


\subsection{The maintenance work memory}\index{maintenance work memory}
The maintenance work memory is set using the GUC parameter 
maintenance\_work\_mem and follow the same rules of work\_mem. This memory 
segment is allocated per user and is used for the maintenance operations 
like VACUUM or REINDEX. As usually this kind of operations happens on one 
relation at time, this parameter can be safely set to a bigger value than 
work\_mem.

\subsection{The temporary memory}
\label{sub:TEMPBUF}
The temporary memory is set using the GUC parameter temp\_buffers. This is the 
amount of memory per user for the temporary table creation before the disk is 
used. Same as for the work memory and the maintenance work memory it's possible 
to change the value for the current session but only before any temporary table 
creation. After this the parameter cannot be changed anymore.


\section{The data area}
\label{sec:PGDATA}\index{data area}
As seen before the data storage area is initialized by initdb \index{initdb}.
Its structure didn't change too much from the old fashioned 7.4.
In this section we'll take a look to the various subdirectories and how their 
usage can affect the performances. 


\subsection{base}\index{data area,base}
\label{sub:BASE}
As the name suggests, the base directory contains the database files. Each 
database have a dedicated subdirectory, named after the internal database's 
object id.
A freshly initialised data directory shows only three 
subdirectories in the base folder.

Those corresponds to the two template databases,template0 and template1, plus 
the postgres database. Take a look to chapter \ref{cha:LOGICLAY} for more 
informations.

The numerical directories contains various files, also with the numerical name 
which are actualy the database's relations, tables and indices. 

The relation's name is set initially from the relation's object id. Any file 
altering operation like VACUUM FULL or REINDEX, will generate a new file 
with a different name. To find out the real relation's file name the 
relfilenode inside the pg\_class system table  must be queried.

\subsection{global}\index{data area,global}
The global directory contains all the cluster wide relations.
In addition there's the very critical  control file mentioned in 
\ref{sub:PGCONTROLDATA} \index{control file}.

This small file is big exactly one database block, usually 8192 bytes, and 
contains critical informations for the cluster. 
With a corrupted control file the instance cannot start. 
The control file is written usually when a checkpoint occurs.

\subsection{pg\_xlog}\index{data area,pg\_xlog}
This is the most important and critical directory, for the performances and for 
the reliability. 

The directory contains the transaction's logs, \index{wal files} named wal 
file. Each file is usually 16 Mb and contains all the data blocks changed during 
the database activity. 
The blocks are written first on this not volatile area to ensure the cluster's 
recovery in case of cras. The data blocks are then written later to the 
corresponding data files. If the cluster's shutdown is not clean then the wal 
files are replayed during the startup process from the last known consistent 
location read from control file.

In order to ensure good performance this location should stay on a dedicated 
device. 

\subsection{pg\_clog}\index{data area, pg\_clog}
This directory contains the committed transactions in small 8k files, except 
for the serializable transactions.
The the files are managed by the cluster and the amount is related with 
the two GUC parameters autovacuum\_freeze\_max\_age and 
vacuum\_freeze\_table\_age.
Increasing the values for the two parameters the pg\_clog must store the commit 
status to the ``event horizon'' of the oldest frozen transaction id. 
More informations about vacuum and the maintenance are in 
the chapter \ref{cha:MAINTENANCE}.

\subsection{pg\_serial}\index{data area, pg\_serial}
Same as pg\_clog this directory stores the informations about the commited 
transactions in serializable transaction isolation level.

\subsection{pg\_multixact}\index{data area, pg\_multixact}
Stores the informations about the multi transaction status, used generally for  
the row share locks.

\subsection{pg\_notify}\index{data area, pg\_notify}
Stores informations about the LISTEN/NOTIFY operations.

\subsection{pg\_snapshots}\index{data area, pg\_snapshots}
This directory is used to store the exported snapshots. From the version 9.2 
PostgreSQL offers the transaction's snapshot export where one session can 
open a transaction and export a consistent snapshot. This way different session 
can access the snapshot and read all togheter the same consistent data 
snapshot. This feature is used, for example, by pg\_dump for the parallel 
export.

\subsection{pg\_stat}\index{data area, pg\_stat}
This directory contains the permanent files for the statistic subsystem. 

\subsection{pg\_stat\_tmp}\index{data area, pg\_stat\_tmp}
This directory contains the temporary files for the statistic subsystem. 
As this directory is constantly written, is very likely to become an 
IO bottleneck. Setting the GUC parameter stats\_temp\_directory to a ramdisk 
speeds can improve the database performances.


\subsection{pg\_subtrans}\index{data area, pg\_subtrans}
Stores the subtransactions status data. 

\subsection{pg\_twophase}\index{data area, pg\_twophase}
Stores the two phase commit data. The two phase commit allows the transaction 
opening independently from the session. This way even a different session can 
commit or rollback the transaction later.

\subsection{pg\_tblspc}\index{data area, pg\_tblspc}
\label{sub:TABLESPACE}
The directory contains the symbolic links to the tablespace locations.
A tablespace is a logical name pointing a physical location. As from PostgreSQL 
9.2 the location is read directly from the symbolic link. This make possible 
to change the tablespace's position simply stopping the cluster, moving the 
data files in the new location, creating the new symlink and starting the 
cluster.
More informations about the tablespace management in the 
chapter \ref{cha:LOGICLAY}. 




\chapter{The logical layout}
\label{cha:LOGICLAY}\index{Logical layout}
In this we'll take a look to the PostgreSQL logical layout.
We'll start with the connection process. Then we'll see the logical relations like tables, indices
and views. The chapter will end with the tablespaces and the MVCC.

\section{The connection}
When a client starts a connection to a running cluster, the process pass through few steps. \newline

The first connection's stage is the check using the host based authentication. The cluster scans the
pg\_hba.conf file searching a  match for the connection's parameters. Those are, for example, the
client's host, the user etc. The host file is usually saved inside the the data area alongside the
configuration file postgresql.conf. The pg\_hba.conf is read from the top to the bottom and the
first matching row for the client's parameters is used to determine the authentication method to
use. If PostgreSQL reaches the end of the file without match the connection is refused.\newline

The pg\_hba.conf structure is shown in \ref{tab:PGHBA}

\begin{table}[H]
  \begin{tabular}{ccccc}
    Type & Database & User & Address & Method \\ 
    \hline
    local & name & name & ipaddress/network mask & trust\\
    host & * & * & host name & reject\\
    hostssl & &  &  & md5\\
    hostnossl & &  &  & password \\
    & & &  & gss \\
    & & &  & sspi \\
    & & &  & krb5 \\
    & & &  & ident \\
    & & &  & peer \\
    & & &  & pam \\
    & & &  & ldap \\
    & & &  & radius \\
    & & &  & cert \\
  \end{tabular}
  \caption{\label{tab:PGHBA}pg\_hba.conf}
\end{table}

The column type specifies if the connection is local or host. The former is when the connection is
made using a socket. The latter when the connection uses the network. It's also possible to
specify if the host connection should be secure or plain using hostssl and hostnossl.\newline

The Database and User columns are used to match specific databases and users.\newline

The column address have sense only if the connection is host, hostssl or hostnossl. The value can
be an ip address plus the network mask. Is also possible to specify the hostname. There is the
full support for ipv4 and ipv6.

The pg\_hba.conf's last column is the authentication method for the matched row. The action to
perform after the match is done. PostgreSQL supports many methods ranging from the plain password
challenge to kerberos.\newline

We'll now take a look to the built in methods.

\begin{itemize}
 \item \textbf{trust}: The connection is authorised without any further action. Is quite useful 
if the password is lost. Use it with caution.

\item \textbf{peer}: The connection is authorised if the OS user matches the 
database user. It's useful for the local connections. 

\item \textbf{password}: The connection establishes if the connection's user and the password
matches with the values stored in the pg\_shadow system table. This method sends the password in
clear text. Should be used only on trusted networks.

\item \textbf{md5}: This method is similar to password. It uses a better security encoding the
passwords using the md5 algorithm. Because md5 is deterministic, there is pseudo random
subroutine which prevents to have the same string sent over the network.

\item \textbf{reject}: The connection is rejected. This method is very useful to keep the sessions
out of the database. e.g. maintenance requiring single user mode.

\end{itemize}

When the connection establishes the postgres main process forks a new backend process attached to
the shared buffer. The fork process is expensive. This makes the connection a potential
bottleneck. Opening new connections can degrade the operating system performance and eventually
produce zombie processes. Keeping the connections constantly connected maybe is a reasonable fix.
Unfortunately this approach have a couple of unpleasant side effects.\newline

Changing any connection related parameter like the max\_connections, requires a cluster restart.
For this reason planning the resources is absolutely vital. For each connection present in
 max\_connections the cluster allocates 400 bytes of shared memory. For each connection established 
the cluster allocates a per user memory area wich size is determined by the parameter
work\_mem.\newline

For example let's consider a cluster with a shared\_buffer set to 512 MB and the work\_mem 
set to 100MB. Setting the max\_connections to only 500 requires a potentially 49 GB of total memory
if all the connections are in use. Because the work\_mem can affects the performances, its
value should be determined carefully. Being a per user memory any change to work\_mem does not
require the cluster's start but a simple reload.\newline 

In this kind of situations a connection pooler can be a good solutions. The sophisticated
\href{http://www.pgpool.net/}{pgpool}  or the
lightweight \href{http://pgfoundry.org/projects/pgbouncer/}{pgbouncer}  can help to boost the
connection's performance.\newline

By default a fresh data area initialisation listens only on the localhost. The GUC parameter
governing this aspect is listen\_addresses. In order to have the cluster accepting connections from
the rest of the network the values should change to the correct listening addresses specified
as values separated by commas. It's also possible to set it to * as wildcard.

Changing the parameters max\_connections and listen\_addresses require the cluster restart.



\section{Databases}
\label{sec:DATABASES}
Unlikely other DBMS, a PostgreSQL connection requires the database name in the connection string.
Sometimes this can be omitted in psql when this information is supplied in another way.\newline

When omitted psql checks if the environment variable \$PGDATABASE \index{\$PGDATABASE variable} is 
set. If \$PGDATABASE is missing then psql defaults the database name to connection's username. This 
leads to confusing error messages. For example, if we have a username named test but not a database 
named test the connection will fail even with the correct credentials.

\begin{verbatim}
postgres@tardis:~$ psql -U test -h localhost
Password for user test: 
psql: FATAL:  database "test" does not exist
\end{verbatim}

This error appears because the pg\_hba.conf allow the connection for any database. Even for a not
existing one. The connection is then terminated when the backend ask to connect to the database 
named test which does not exists.\newline

This is very common for the new users. The solution is incredibly simple because in a PostgreSQL 
cluster there are at least three databases. Passing the name template1 as last parameter will do 
the trick.

\begin{verbatim}
postgres@tardis:~$ psql -U test -h localhost template1
Password for user test: 
psql (9.3.4)
SSL connection (cipher: DHE-RSA-AES256-SHA, bits: 256)
Type "help" for help.
\end{verbatim}

When the connection is established we can query the system table pg\_database to get the 
cluster's database list. 

\begin{lstlisting}[style=pgsql]
template1=> SELECT datname FROM pg_database;
    datname    
---------------
 template1
 template0
 postgres
(3 rows)

\end{lstlisting}

Database administrators coming from other DBMS can be confused by the postgres database.
This database have nothing special. Its creation was added since the version 8.4 because it was 
useful to have it. You can just ignore it or use it for testing purposes. Dropping the postgres 
database does not corrupts the cluster. Because this database is often used by third party 
tools before dropping it check if is in use in any way.\newline

The databases template0 and template1 \index{template1 database} \index{template0 
database} like the name suggests are the template databases. A template database \index{template 
database} is used to build new database copies via the physical file copy. 

When initdb initialises the data area the database template1 is populated with the correct
references to the WAL records, the system views and the procedural language PL/PgSQL. When
this is done the database template0 and the postgres databases are then created using the template1
database.

The database template0 doesn't allow the connections. It's main usage is to rebuild the
database template1 if it gets corrupted or for creating databases with a character encoding/ctype, 
different from the cluster wide settings. 
\index{CREATE DATABASE}

\begin{lstlisting}[style=pgsql]
postgres=# CREATE DATABASE db_test WITH ENCODING 'UTF8' LC_CTYPE 'en_US.UTF-8';
ERROR:  new LC_CTYPE (en_US.UTF-8) is incompatible with the LC_CTYPE of the 
template database (en_GB.UTF-8)
HINT:  Use the same LC_CTYPE as in the template database, or use template0 as 
template.

postgres=# CREATE DATABASE db_test WITH ENCODING 'UTF8' LC_CTYPE 'en_US.UTF-8' 
TEMPLATE template0;
CREATE DATABASE
postgres=# 

\end{lstlisting}

If the template is omitted the CREATE DATABASE statement will use template1 by default. 


A database can be renamed or dropped with ALTER DATABASE and DROP DATABASE \index{ALTER 
DATABASE}\index{DROP DATABASE} statements. Those operations require the exclusive access to the 
affected database. If there are connections established the drop or rename will fail.

\begin{lstlisting}[style=pgsql]
postgres=# ALTER DATABASE db_test RENAME TO db_to_drop;
ALTER DATABASE

postgres=# DROP DATABASE db_to_drop;
DROP DATABASE

\end{lstlisting}




\section{Tables}\index{Tables}
\label{sec:TABLES}
In our top down approach to the PostgreSQL's logical model, the next step is the relation.
In the PostgreSQL jargon a relation is an object which carries the data or the way to
retrieve the data. A relation can have a physical counterpart or be purely logical. We'll take a 
look in particular to three of them starting with the tables.\newline

A table is the fundamental storage unit for the data. PostgreSQL implements many kind of tables
with different levels of durability. A table is created using the SQL command CREATE TABLE. The data
is stored into a table without any specific order. Because the MVCC implementation a row update can
change the row's physical position. For more informations look to \ref{sec:MVCC}. PostgreSQL
implements three kind of tables.

\subsection{Logged tables}\index{Logged tables}
By default CREATE TABLE creates a logged table. This kind of table implements the durability
logging any change to the write ahead log. The data pages are loaded in the shared buffer and any
change to them is logged first to the WAL. The consolidation to the the data file happens later. 

\subsection{Unlogged tables}\index{Unlogged tables}
\label{sub:UNLOGGEDTABLES}
An unlogged table have the same structure like the logged table. The difference is such kind of 
tables are not crash safe. The data is still consolidated to the data file but the pages modified 
in memory do not write their changes to the WAL. The main advantage is the write operations which 
are considerably faster at the cost of the data durability. The data stored into an ulogged table
should be considered partially volatile. The database will truncate those tables when the crash
recovery occurs. Because the unlogged table don't write to the WAL, those tables are not accessible 
on a physical standby. 

\subsection{Temporary tables}\index{Temporary tables}
A temporary table is a relation which lives into the backend's local memory. When the connection 
ends the table is dropped. Those table can have the same name for all the sessions because 
they are completely isolated. If the amount of data stored into the table is lesser than 
the temp\_buffers value the table will fit in memory with great speed advantage. Otherwise the 
database will create a temporary relation on disk. The parameter temp\_buffers can be altered for 
the session but only before the first temporary table is created. 


\subsection{Foreign tables}\index{Foreign tables}
The foreign tables were first introduced with PostgreSQL 9.1 as read only relations, improving 
considerably the DBMS interconnection's capability. A foreign table works exactly like a local table 
using a foreign data
wrapper to interact with the foreign data source.\newline

There are many different foreign data wrappers available for very exotic data sources. With the 
PostgreSQL 9.3 there is the postgres\_fdw and the the foreign tables are writable. In particular 
the postgres\_fdw implementation is similar to old dblink module with a more efficient performance 
management and the connection's caching.

\section{Table inheritance}\index{Table inheritance}
Being more precise, PostgreSQL is an Object Relational Database Management System. Its intenal logic
implements some of the concepts of the object oriented programming. The relations are also called 
classes and the table's columns attributes. \newline

The table inheritance is a logical relationship between a parent table and one or more child 
tables. The child tables inherit the parent's attribute structure but not the physical storage. 

\begin{lstlisting}[style=pgsql]
db_test=#CREATE TABLE t_parent
                      (
                          i_id_data     integer,
                          v_data        character varying(300)
                      );

CREATE TABLE                     

db_test=#CREATE TABLE t_child_01
                      
                      ()
             INHERITS (t_parent)
                      ;                      
db_test=# \d t_parent
            Table "public.t_parent"
  Column   |          Type          | Modifiers 
-----------+------------------------+-----------
 i_id_data | integer                | 
 v_data    | character varying(300) | 
Number of child tables: 1 (Use \d+ to list them.)

db_test=# \d t_child_01 
           Table "public.t_child_01"
  Column   |          Type          | Modifiers 
-----------+------------------------+-----------
 i_id_data | integer                | 
 v_data    | character varying(300) | 
Inherits: t_parent

\end{lstlisting}

The inheritance can be defined at creation time or later. It's also possible to create a completely 
new table and create the inheritance with another one. In this case the child structure must match 
exactly the parent.

\begin{lstlisting}[style=pgsql]

db_test=# ALTER TABLE t_child_01 NO INHERIT t_parent;
ALTER TABLE
db_test=# ALTER TABLE t_child_01 INHERIT t_parent;
ALTER TABLE

\end{lstlisting}

Because the physical storage is not shared the unique constraints aren't globally enforced on the 
inheritance tree and this prevents the foreign keys to refer efficiently the inherited tables. This 
makes the table partitioning tricky.

\section{Indices}
An index is a structured relation where the indexed values are stored. The index entries are 
associated with pointers to the corresponding table's pages. Because PostgreSQL have a cost based 
optimiser, having an an index doesn't guarantee its usage. Reading an index page requires a random
disk seek. The cost for random is estimated by default four times more than a sequential page read
used by the sequential table scan. Those values can be adjusted using the two GUC parameters
seq\_page\_cost and random\_page\_cost, respectively the arbitrary cost of a table's page and an
index page. \newline

Designing the indices is a complex task. An unused index adds overhead and slowness to the write
operations with no benefit. Querying system views like the pg\_stat\_all\_indexes will give us
vital informations about the indices usage.

This simple finds all the indices in the public schema never used from the last statistics 
reset.

\begin{lstlisting}[style=pgsql]
SELECT
        schemaname,
        relname,
        indexrelname,
        idx_scan
FROM
         pg_stat_all_indexes
WHERE
                schemaname='public'
        AND     idx_scan=0
;

\end{lstlisting}


PostgreSQL implements many index type. Unfortunately the bitmap index, useful in the datawarehouse,
is not present. There is a plan node which can emulate partially the bitmap indices. This is
done with a sequential read over the index followed by a table's read for rematch. The tuples
are found using a bitmap generated from the index's sequential read.\newline 


The keyword USING specifies the index type at create time.
\begin{lstlisting}[style=pgsql]
 CREATE INDEX idx_test ON t_test USING hash (t_contents);
\end{lstlisting}

If omitted then the index type defaults to the B-tree.

The index maintenance is a delicate matter, the argument is described in depth
in \ref{cha:MAINTENANCE}.


\subsection{b-tree}
The general purpose B-tree\index{index,b-tree} index implements the Lehman and Yao's
high-concurrency B-tree management algorithm. The B-tree can handle equality and range queries 
returning ordered data. The indexed values are stored into the index pages with pointers to the
table's pages. The index is not TOASTable, this limit the max length for an index entry to 1/3 of
the page size. \newline

\subsection{hash}
The hash indices\index{index,hash} can handle only equality and are not WAL logged. Their changes
are not replayed if the crash recovery occurs and do not propagate to the standby servers.\newline

\subsection{GiST}
The GiST indices\index{index,GiST} are the Generalised Search Tree. The GiST is a collection of
indexing strategies organised under a common infrastructure. They can implement arbitrary indexing
schemes like B-trees, R-trees  or other. The operator classes shipped with PostgreSQL are for the
geometrical data with two elements and for the nearest-neighbour searches. The GiST indices do not
perform an exact match. The false positives are removed with second rematch on the
table's data.\newline

\subsection{GIN}
The GIN indices \index{index,GIN} are the Generalised Inverted Indices. This kind of index
is optimised for indexing the composite data types or vectors like the full text search elements.
The GIN are exact indices, when scanned the returned set doesn't require recheck.



\section{Views}
\index{views}
\label{sec:VIEWS}
A view is a relation storing the logical representation of a query. At create time all the objects
involved in the view definition are translated in the internal representation. Any wildcard is
expanded.

An example will explain better this concept. Let's consider a simple table populated with the
generate\_series() function.

\begin{lstlisting}[style=pgsql]


CREATE TABLE t_data 
        ( 
                i_id            serial,
                t_content       text
        );

ALTER TABLE t_data 
ADD CONSTRAINT pk_t_data PRIMARY KEY (i_id);


INSERT INTO t_data
        (
                t_content
        )
SELECT
        md5(i_counter::text)
FROM
        (
                SELECT
                        i_counter
                FROM
                        generate_series(1,200) as i_counter
        ) t_series;

CREATE OR REPLACE VIEW v_data 
AS 
  SELECT 
          *
  FROM 
        t_data;


\end{lstlisting}

Running a select from the view or the table returns the same columns and rows.
The stored definition in pg\_views does not have any wildcard though.


\begin{lstlisting}[style=pgsql]
db_test=# \x
db_test=# SELECT * FROM pg_views where viewname='v_data';
-[ RECORD 1 ]--------------------
schemaname | public
viewname   | v_data
viewowner  | postgres
definition |  SELECT t_data.i_id,
           |     t_data.t_content
           |    FROM t_data;


\end{lstlisting}

Adding a new column to the t\_data table will break the equality between the table and the view.

\begin{lstlisting}[style=pgsql]
 ALTER TABLE t_data ADD COLUMN d_date date NOT NULL default now()::date;
 
 db_test=# SELECT * FROM t_data LIMIT 1;
 i_id |            t_content             |   d_date   
------+----------------------------------+------------
    1 | c4ca4238a0b923820dcc509a6f75849b | 2014-05-21
(1 row)


db_test=# SELECT * FROM v_data LIMIT 1;
 i_id |            t_content             
------+----------------------------------
    1 | c4ca4238a0b923820dcc509a6f75849b
(1 row)


 
\end{lstlisting}

In order to have the view in sync with the table we need to refresh it using the CREATE OR REPLACE
VIEW statement.

\begin{lstlisting}[style=pgsql]
 CREATE OR REPLACE VIEW v_data 
AS 
  SELECT 
        *
  FROM 
        t_data;
        
db_test=# SELECT * FROM v_data LIMIT 1;
 i_id |            t_content             |   d_date   
------+----------------------------------+------------
    1 | c4ca4238a0b923820dcc509a6f75849b | 2014-05-21
(1 row)

\end{lstlisting}

PostgreSQL implements the views using the the objects identifiers. They never invalidate when
the objects are renamed. \newline

The CREATE OR REPLACE statement can update the view definition only if the column 
list adds new attributes in the end. When one or more view are pointing a relation this cannot be
dropped unless the clause CASCADE is used. Because the dependencies can be very complicated, this
approach is quite dangerous. The best is querying on the pg\_depend table and find out the full
list of dependencies.\newline

Because a view is a logical short cut to a stored SQL, PostgreSQL will run the query in the same
way as it was sent from the client. This excludes the network overhead.

A view can refer other views becoming a potential cause of performance regression. Because any view
have a specific execution plan, mixing different plans can confuse the planner causing
poor performance.

To avoid this it's a a good practice using the naming  v\_ to distinguish the views from the
tables. In \ref{cha:COUPLETHINGS} there is an example for the self explanatory database schema.

\index{view, updatable}
PostgreSQL from the version 9.3 supports the updatable views. This feature is limited just to the
simple views. A view is simple  if have the following features.

\begin{itemize}


 \item   Have exactly one entry in its FROM list, which must be a table or another updatable view.

 \item Does not contain WITH, DISTINCT, GROUP BY, HAVING,LIMIT, or OFFSET clauses at the top level.

 \item  Does not contain set operations (UNION, INTERSECT or EXCEPT) at the top level
 
 \item   All columns in the view's select list must be simple references to columns of the
underlying relation. They cannot be expressions, literals or functions. System columns cannot be
referenced, either.

 \item   Columns of the underlying relation do not appear more than once in the view's select list.

 \item   Does not have the security\_barrier property.

\end{itemize}

A complex view can be updatable using the triggers or rules.\newline

\index{view, materialised}
Another feature introduced by the 9.3 is the materialised views. This is a physical snapshot of the
saved SQL and can be refreshed with the statement REFRESH MATERIALIZED VIEW.  


\section{Tablespaces}\index{tablespaces,logical}
\label{sub:TBS-LOGICAL}
A tablespace\index{tablespace} is a logical name pointing to a physical location. 
This feature was introduced with the release 8.0 and its implementation did not change since
then. From the version 9.2 was introduced the function pg\_tablespace\_location(tablespace\_oid)
which gets the tablespace's location from the OID making the dba life easier.

When a new relation is created without the tablespace specification this defaults to the GUC
parameter default\_tablespace or to the database's default tablespace. 
On a fresh initialised cluster the initial value for the default tablespace is pg\_default which
points to to the directory \$PGDATA/base . Alongside with pg\_default there is the pg\_global as
well. This tablespace is reserved for the shared objects and points to the directory
\$PGDATA/global.

Creating a new tablespace is very simple. The physical location must be owned by the postgres
process user. 

For example let's create a tablespace pointing to the folder /var/lib/postgresql/pg\_tbs/ts\_test
with the name ts\_test.

\begin{lstlisting}[style=pgsql]
CREATE TABLESPACE ts_test 
OWNER postgres
LOCATION '/var/lib/postgresql/pg_tbs/ts_test' ;

\end{lstlisting}

Only superusers can create tablespaces. The OWNER clause is optional, if 
omitted the tablespace's owner defaults to the database user logged in. The tablespaces are cluster
wide and are listed into the pg\_tablespace system table.\newline

Creating a new relation into the tablespace ts\_test requires the TABLESPACE clause followed by the
tablespace name.

\begin{lstlisting}[style=pgsql]
CREATE TABLE t_ts_test
        (
                i_id serial,
                v_value text
        )
TABLESPACE ts_test ;

\end{lstlisting}

A relation can be moved from a tablespace to another using the ALTER command. For example let's
move the table t\_ts\_test to the  pg\_default tablespace.

\begin{lstlisting}[style=pgsql]
ALTER TABLE t_ts_test SET TABLESPACE pg_default;
\end{lstlisting}

Changing the relation's tablespace is transaction safe but requires an exclusive lock on the
affected relation. The lock prevents accessing the relation's data for the time required by the
move.If the relation have a significant size this should be considered carefully. Because the
exclusive lock conflicts with the locks issued by the backup, it's not possible to change the
relation's tablespace when pg\_dump is running.\newline


A tablespace can be removed with DROP TABLESPACE command. The tablespace must be empty before the
drop. There's no CASCADE clause to have the tablespace's contents dropped with the tablespace.

\begin{lstlisting}[style=pgsql]
postgres=# DROP TABLESPACE ts_test;
ERROR:  tablespace "ts_test" is not empty

postgres=# ALTER TABLE t_ts_test SET TABLESPACE pg_default;
ALTER TABLE
postgres=# DROP TABLESPACE ts_test;
DROP TABLESPACE

\end{lstlisting}


The tablespace feature adds flexibility to the space management. Even if not sophisticated like
other DBMS this implementation, combined with a careful design can improve sensibly the
performances.\newline

In \ref{sub:TBS-PHYSICAL} we'll take a look to the how PostgreSQL implements the tablespaces on the
physical side.



\section{Transactions}
\label{sec:TRANSACTION}
\index{transactions}
The atomicity the consistency and the isolation is implemented in PostgreSQL via the
MVCC\index{MVCC}. This is the acronym for Multi Version Concurrency Control\index{Multi Version
Concurrency Control}. This solution offers high efficiency in the concurrent user access for read
and write queries.\newline 

When a transaction starts a write operation receives an identifier, the XID \index{XID}. This is
a 32 bit quantity and is used to determine the transaction's visibility. All the transactions with
XID lesser than the current XID and committed are considered in the past and then visible. All the
transactions with XID greater than the current XID are in the future and not
visible. The not committed transactions are considered invisible as well.\newline

This comparison happens at tuple level using two system fields xmin and xmax. When a transaction
creates a new tuple the xmin is set with the transaction's xid. This field is also referred as the
insert's transaction id. When a tuple is deleted then the xmax value is updated to the transaction's
xid. The tuple when deleted is left in place, ensuring the read consistency for any transaction
which should see the previous version. The last tuple's version is a live tuple. The tuples which
versions are outdated are called dead tuples.\newline

There is no field for the update. In PostgreSQL the update is a a complete new insert. The update's
transaction id is used either for the new tuple's xmin and the old tuple's xmax. The dead tuples are
removed by VACUUM if no longer needed by existing transactions. In \ref{sec:TUPLES}
there is the tuple structure's description.\newline

Alongside with xmin and xmax there are two other system fields, the cmin and cmax. Their data
type is the command id, CID. Those fields are similar to xmin and xmax. They are used to track the
internal transaction's commands in order to avoid the command to be executed on the same
tuple multiple times. This solve the database's Halloween Problem described there
\href{http://en.wikipedia.org/wiki/Halloween_Problem}{
http://en.wikipedia.org/wiki/Halloween\_Problem}.\newline

The SQL standard defines four level of transaction's isolation levels where 
some phenomena are permitted or forbidden.
\index{transactions, isolation levels}

\begin{itemize}
 \item \textbf{dirty read}, when a transaction reads the data written by a concurrent 
uncommitted transaction

\item \textbf{nonrepeatable read} when a transaction re reads the same data and finds it changed by
another transaction which has committed since the initial read

\item \textbf{phantom read} when a transaction re executes a query returning a set 
of rows for a search condition and finds that results are changed by a recently committed 
transaction

\end{itemize}

The table \ref{tab:TRNISOLATION} shows the transaction's isolation levels with the phenomena
possible. In PostgreSQL it's possible to set all the four isolation levels but only the three
strictest levels are supported. Setting the  isolation level to read uncommited is
equivalent to set it to the read committed.

\begin{table}[H]
  \begin{tabular}{cccc}
    Isolation Level & Dirty Read    &    Nonrepeatable Read   &   Phantom 
Read\\ 
    \hline
    Read uncommitted  &  Possible    &    Possible     &   Possible\\
    Read committed    &  Not possible &  Possible     &   Possible\\
    Repeatable read   &  Not possible  & Not possible  &  Possible\\
    Serializable      &  Not possible  & Not possible   & Not possible\\
  \end{tabular}
  \caption{\label{tab:TRNISOLATION}SQL Transaction isolation levels}
\end{table}

The default the global isolation level is read committed. However is possible to change the
isolation level per session with the command:
\begin{lstlisting}[style=pgsql]
SET TRANSACTION ISOLATION LEVEL { SERIALIZABLE | REPEATABLE READ | READ 
COMMITTED | READ UNCOMMITTED }; 
\end{lstlisting}

Changing the default transaction isolation for the entire cluster is made using the GUC parameter
transaction\_isolation.


\subsection{Snapshot exports}
\label{sub:SNAPEXPORT}\index{transactions, snapshot export}
Since PostgreSQL 9.2 are supported the transaction's snapshot exports. A session with an open 
transaction, can export its consistent snapshot to any other session. The snapshot remains valid 
meanwhile the transaction is open. Using this functionality offers a way to run multiple backends 
on a consistent data set frozen in time. This feature resulted in the brilliant parallel export in 
the 9.3's pg\_dump as described in \ref{sec:PGDUMPINT}.\newline

In the following example, let's consider the table created in \ref{sec:VIEWS}. We'll first start an 
explicit transaction and then we'll export the current snapshot.

\begin{lstlisting}[style=pgsql]
postgres=# BEGIN TRANSACTION ISOLATION LEVEL REPEATABLE READ;
BEGIN
postgres=# SELECT pg_export_snapshot();
 pg_export_snapshot 
--------------------
 00001369-1
(1 row)

postgres=# SELECT count(*) FROM t_data;
 count 
-------
   200
(1 row)

\end{lstlisting}

We are first starting a transaction with the REPEATABLE READ isolation level. The second 
statement exports the current snapshot using the function pg\_export\_snapshot(). Finally we are 
checking with a simple row count the table t\_data have data inside.\newline

We can now login with in a different session and delete all the rows from the t\_data table.

\begin{lstlisting}[style=pgsql]
postgres=# DELETE FROM t_data;
DELETE 200
postgres=# SELECT count(*) FROM t_data;
 count 
-------
     0
(1 row)

\end{lstlisting}

With the table now empty let's import the snapshot exported by the first backend.

\begin{lstlisting}[style=pgsql]
postgres=# BEGIN TRANSACTION ISOLATION LEVEL REPEATABLE READ;
BEGIN
postgres=# SET TRANSACTION SNAPSHOT '00001369-1';
SET
postgres=# SELECT count(*) FROM t_data;
 count 
-------
   200
(1 row)

\end{lstlisting}

The function pg\_export\_snapshot saves the current snapshot returning the text string which 
identifies the snapshot. Passing the string to clients that want to import the snapshot gives to 
independent sessions a single consistent vision. The import is possible only until the end of the 
transaction that exported it. The export is useful only in the READ COMMITTED transactions, 
because the REPEATABLE READ and higher isolation levels use the same snapshot within their 
lifetime. 


\chapter{Data integrity}
There's just one thing worse than losing the database. Having the data set full of rubbish. The 
data integrity has been part of PostgreSQL since the beginning. It offers various levels of 
strength ensuring the data is clean and consistent. In this chapter we'll have a brief look to the 
various constraints available. The PostgreSQL's constraints can be grouped in two kind. The table 
constraints and the column constraints. 

The table constraints are defined on the table's definition after the field's list. The column 
constraints appear in the field's definition after the data type. Usually for the primary keys and 
the unique keys the definition is  written as table constraint. 

The constraint applies the enforcement to any table's row without exclusion. When creating a table 
constraint on a fully populated table the data is validated first. Any validation error aborts the 
constraint creation. However, the foreign keys and check constraints accept the clause NOT VALID. 
With this clause the database  assumes the data is valid and skips the validation. The cration is 
almost immediate. The new constraint is then enforced only for the new data. When using this 
option  the data must be consistent.

\section{Primary keys} 
A primary key is the unique row identifier. Having this constraint enforced ensures the row can be 
addressed directly using the key value. A primary key can be enforced on a single or multi column. 
The data aspect must be unique with the strictest level. That means the NULL values are not 
permitted in columns participating to the primary key. When creating a primary key this 
implicitly adds a new unique index on the affected fields. In order to avoid the exclusive lock 
on the affected  table the unique index can be built before the primary using the CONCURRENTLY 
clause and then used in the primary key definition as shown in \ref{sec:REINDEX}. Using the primary 
key is the fastest way to access the table's contents.\newline

There is the primary key definition as table and column constraint.\newpage

\begin{lstlisting}[style=pgsql]

--PRIMARY KEY AS TABLE CONSTRAINT
CREATE TABLE t_table_cons
        (
                i_id            serial,
                v_data          character varying (255),
                CONSTRAINT pk_t_table_cons PRIMARY KEY (i_id)
        )
;


--PRIMARY KEY AS COLUMN CONSTRAINT
CREATE TABLE t_column_cons
        (
                i_id            serial PRIMARY KEY,
                v_data          character varying (255)
        )
;
 

\end{lstlisting}

With the table's constraint definition is possible to specify the constraint name and to have a 
multi column constraint. When writing a multi column constraint the participating columns should be 
listed separate by commas.\newline

The most common primary key implementation, and probably the best, is to have a serial column as 
primary key. A serial field is short for integer NOT NULL which default value is associated to the 
nextval for an auto generated sequence. Because the sequence have its upper limit to the bigint 
upper limit, this ensures the data does not wraps in the table's lifetime. In the case the primary 
key is expected to reach the value of 2,147,483,647 the type of choice should be bigserial rather 
serial. This will create the primary key's field as bigint which upper limit is 
9,223,372,036,854,775,807.\newline

However it's still possible to alter the field later in order to match the new requirements. 
Because changing the data type requires a complete table's rewrite, any view referencing the 
affected column will abort the change. \newline 

Here's the t\_data's type change output with the client message level set to debug3.

\begin{lstlisting}[style=pgsql]
postgres=# ALTER TABLE t_data ALTER COLUMN i_id SET DATA TYPE  bigint; 
DEBUG:  StartTransactionCommand
DEBUG:  StartTransaction
DEBUG:  name: unnamed; blockState:       DEFAULT; state: INPROGR, xid/subid/cid: 0/1/0, nestlvl: 1, 
children: 
DEBUG:  ProcessUtility
DEBUG:  drop auto-cascades to index pk_t_data
DEBUG:  building index "pg_toast_43551_index" on table "pg_toast_43551"
DEBUG:  rewriting table "t_data"
DEBUG:  building index "pk_t_data" on table "t_data"
DEBUG:  drop auto-cascades to type pg_temp_28448
DEBUG:  drop auto-cascades to type pg_temp_28448[]
DEBUG:  drop auto-cascades to toast table pg_toast.pg_toast_28448
DEBUG:  drop auto-cascades to index pg_toast.pg_toast_28448_index
DEBUG:  drop auto-cascades to type pg_toast.pg_toast_28448
DEBUG:  CommitTransactionCommand
DEBUG:  CommitTransaction
DEBUG:  name: unnamed; blockState:       STARTED; state: INPROGR, xid/subid/cid: 9598/1/21 (used), 
nestlvl: 1, children: 
ALTER TABLE

\end{lstlisting}


Dealing with a big amount of data presents also the problem to have enough space for fitting twice 
of the original table plus the downtime caused by the exclusive lock on the affected relation. A 
far better approach is to add a new bigint NULLable column without default value. Setting up a 
trigger for the inserts will keep in sync the new values with the original primary key. Then an 
update procedure will set the value for the rows. This should run in small batches to avoid to 
overfill the pg\_xlog directory with long running transactions. When everything is in place the new 
column could then become NOT NULL and a unique index will finally enforce the uniqueness for the 
new 
field. \newline

The primary key can then be dropped and recreated using the new unique index. This is permitted 
only if there's no foreign key referencing the field. In this case a multi drop and create 
statement is required. The final cleanup should include the trigger's drop and the old primary key 
removal. Any view using the old primary key should be rebuilt before the drop.

\section{Unique keys}
The unique keys are similar to the primary keys. They enforce the uniqueness using an implicit 
index but they allow the presence of NULL values. Their usage is for enforcing uniqueness on 
columns not used as primary key. Similar to the primary key the unique constraints are based on a 
unique index. In fact there's little difference between the unique index and the unique key except 
the presence of the latter in the system table pg\_constraint. 


\section{Foreign keys}
\section{Check constraints}
\section{Exclusion constraints}

\section{Not null}
\section{Type enforcement}
\section{Custom data types}


\chapter{The physical layout}
\label{cha:PHYLAY}\index{Physical layout}
After looking to the logical structure we'll now dig into PostgreSQL's physical structure. 
We'll start with the top layer, looking into the data area. We'll take a look first to the 
data files and how they are organised. Then we'll move inside them, where the data pages 
and the fundamental storage unit, the tuples, are stored. A section is dedicated to the 
TOAST tables. The chapter will end with the physical aspect of the tablespaces and the 
MVCC\index{MVCC}.

\section{Data files}\index{Data files}
As seen in \ref{sec:PGDATA} the data files are stored into the \$PGDATA/base directory, 
organised per database object identifier. This is true also for the relations created 
on a different tablespace. Inside the database directories there are many files which 
name is numeric as well. When a new relation is created, the name is set initially to the 
relation's object identifier. The relation's file name can change if any actiont like 
REINDEX or VACUUM FULL is performed on the relation.\newline

The data files are organised in multiple segments, each one of 1 GB and numbered with a 
suffix. However the first segment created is without suffix. Alongside the main data 
files there are some additional forks needed used by PostgreSQL for tracking the data 
visibility and free space.

\subsection{Free space map}\index{Free space map}
The free space map is a segment present alongside the index\index{Index, files} and 
table's data files . It have the same the relation's name with the suffix \_fsm. 
PostgreSQL stores the information of the free space available. 

\subsection{Visibility map}\index{Visibility map}
The table's data file have a visibility map file which suffix is \_vm. PostgreSQL 
tracks the data pages with all the tuples visible to the active transactions. This fork 
is also used for running the index only scans\index{index only scans}.

\subsection{Initialisation fork}\index{Initialisation fork}
The initialisation fork is an empty file used to re initialise the unlogged relations 
when the cluster performs a crash recovery.

\subsection{pg\_class}
When connecting to a database, all the relations inside it are listed in the 
pg\_class\index{pg\_class} system table. The field relfilenode stores the relation's 
filename. The system field oid, which is hidden when selecting with the wildcard *, is 
just the relation's object identifier and should not be used for the physical 
mapping.\newline

However, PostgreSQL have many useful functions which retrieve the information 
using the relation's OID. For example the function pg\_total\_relation\_size(regclass) 
returns the disk space used by the table, including the additional forks and the eventual 
TOAST table, andthe indices. The function returns the size in bytes. Another function, 
the pg\_size\_pretty(bigint), returns a human readable format for better reading.\newline

The pg\_class's field relkind is used to store the relation's kind.

\begin{table}[h]
  \begin{tabular}{cc}
    Value & Relation's kind\\
    \hline
    r  &  ordinary table \\
    i  &  index \\
    S  &  sequence \\
    v  &  view \\
    m  &  materialised view \\
    c  &  composite type \\
    t  &  TOAST table \\
    f  &  foreign table \\
    
  \end{tabular}
  \caption{\label{tab:RELKIND}Relkind values}
\end{table}

\section{Pages}\index{Data pages}
Each datafile is a collection of elements called pages. The default size is for a data 
page is 8 kb. The page size can be changed only recompiling the sources with the 
different configuration and re initialising the data area. Table's pages are also 
known as heap pages\index{Heap pages}. The index pages\index{Index pages} have almost the 
same heap structure except for the special space allocated in the page's bottom. The 
figure \ref{fig:INDEX01} shows an index page structure. The special  space is used 
to store information needed by the relation's structure. For example a B-tree index 
puts in the special space the pointers to the pages below in the B-tree structure.

\begin{figure}[H]
\begin{center}

\includegraphics[scale=0.35]{images/index_page_01.png}

\caption{Index page}
\label{fig:INDEX01} 
\end{center}

\end{figure}

A data page starts with a header of \index{Data pages,header}24 bytes. After the 
header there are the item pointers, which size is usually 4 bytes. Each item 
pointer\index{Item pointers} is an array of pairs composed by the offset and the length 
of the item which ponints the physical tuples in the page's bottom.\newline 

The page header holds the information for the page's generic space management as shown 
in figure \ref{fig:HEADERPAG01}. 


\begin{figure}[H]
\begin{center}

\includegraphics[scale=0.55]{images/header_page_01.png}

\caption{Page header}
\label{fig:HEADERPAG01} 
\end{center}

\end{figure}
\begin{itemize}
 \item \textbf{pd\_lsn} identifies the xlog record for last page's change.  The 
buffer manager uses the  LSN for enforcing the WAL mechanism. A dirty buffer is not 
dumped to the disk until the xlog has been flushed at least as far as the page's LSN.
\item \textbf{pd\_checksum} stores the page's checksum if is enabled.
\item \textbf{pd\_flags} is used to store the page's various flags 
\item \textbf{pg\_lower} is the offset to the start of the free space
\item \textbf{pg\_upper} is the offset to the end of the free space
\item \textbf{pg\_special} is the offset to the start of the special space
\item \textbf{pd\_pagesize\_version} is the page size and the page version packed 
together in a single field. 
\item \textbf{pg\_prune\_xid} is a hint field to determine if the tuple's pruning is 
useful. Is set only on the heap pages.

\end{itemize}

The pd\_checksum \index{Page checksum}field replaces the pd\_tli field present in the page 
header until PostgreSQL 9.2 which was used to track the xlog records across the timeline id. 
\newline 

The page's checksum is a new 9.3's feature which can detects the page corruption. It can be enabled only 
when the data area is initialised with initdb.\newline

The offset fields, pg\_lower, pd\_upper and the optional pd\_special, are 2 bytes long limiting the 
max page size to 32KB.\newline

The field for the page version\index{Page version} was introduced with PostgreSQL 7.3. 
Table \ref{tab:PGPAGEVERSION} shows the page version number for the major versions.

\begin{table}[h]
  \begin{tabular}{cc}
    PostgreSQL version & Page version\\
    \hline
    \textgreater \space 8.3  &  4\\
    8.1,8.2  &  3\\
    8.0  &  2\\
    7.4,7.3  &  1\\
    \textless \space 7.3  &  0\\
    
    
  \end{tabular}
  \caption{\label{tab:PGPAGEVERSION}PostgreSQL page version}
\end{table}

\section{Tuples}\index{Tuples}
\label{sec:TUPLES}
The tuples are the fundamental storage unit in PostgreSQL. They are organised as array of items which kind 
is initially unknown, the datum. Each tuple have a fixed header of 23 bytes as shown in the figure 
\ref{fig:TUPLES01}.\newline

\begin{figure}[H]
\begin{center}

\includegraphics[scale=0.55]{images/tuples_01.png}

\caption{Tuple structure}
\label{fig:TUPLES01} 
\end{center}

\end{figure}

The fields t\_xmin\index{t\_xmin} and t\_xmax\index{t\_xmax} are used to track the tuple's visibility as 
seen in \ref{sec:MVCC}. The field t\_cid\index{t\_cid} is a ``virtual'' field and is used either for cmin 
and cmax. \newline

The field t\_xvac\index{t\_xvac} is used by VACUUM when moving the rows, according with the source code's 
comments in src/include/access/htup\_details.h this field is used only by the old style VACUUM FULL. 
\newline

The field t\_cid\index{t\_cid} is the tuple's physical location identifier. Is composed by a couple of 
integers representing the page number and the tuple's index along the page. When a new tuple is created 
t\_cid is set to the actual row's value. When the tuple is updated the this 
value changes to the new tuple's version location. This field is used in pair with t\_xmax to check if 
the tuple is the last version. The two infomask fields are used to store various flags like the presence of 
the tuple's OID or if the tuple have NULL values. The last field t\_off is used to set the offset to the 
actual tuple's data. This field's value is usually zero if the table doesn't have NULLable fields or is 
created WITHOUT OIDS. If the tuples have the OID and or a NULLable fields, the object identifier and 
a NULL bitmap are stored immediately after the tuple's header. The bitmap if present begins just after the 
tuple's header and consumes enough bytes to have one bit per data column. The OID if present is stored 
after the bitmap and consumes 4 bytes. The tuple's data is a stream of composite data described by the 
composite model stored in the system catalogue. 


\section{TOAST}\index{TOAST}
\label{sec:TOAST}
The oversize attribute storage technique is the PostgreSQL implementation for storing the data 
which overflows the page size. PostgreSQL does not allow the tuples spanning multiple pages. However is 
possible to store large amount of data which is compressed or split in multiple rows in an external 
TOAST table. The mechanism is completely transparent from the user's point of view.\newline

The storage model treats the fixed length, like the integers, and the variable length types, like text, in 
a different way. The fixed length types which cannot produce large data are not processed through the TOAST 
routines. The variable length types are TOASTable if the first 32-bit word of any stored value contains the 
total length of the value in bytes (including itself).

The kind of the TOAST is stored in the first two bits\footnote{On the big-endian architecture those are the 
high-order bits; on the little-endian those are the low-order bits} of the varlena\index{varlena} length 
word. When both bits are zero then the attribute is an unTOASTed data type. In the remaining bits is stored 
the datum size in bytes including the length word.\newline

If the first bit is set then the value have only a single-byte header instead of the four byte header. 
In the remaining bits is stored the total datum size in bytes including the length byte. This scenario 
have a special case uf the remaining bits are all zero. This means the value is a pointer to an out of line 
data stored in a separate TOAST table which structure is shown in figure \ref{fig:TOAST01}.\newline

Finally, whether is the first bit,  if the second bit is set then the corresponding datum is compressed and 
must be decompressed before the use.\newline

Because the TOAST usurps the first two bits of the varlena length word it limits the max stored size to 1 
GB  \begin{math} (2^{30} -1 bytes) \end{math} .

\begin{figure}[H]
\begin{center}

\includegraphics[scale=0.55]{images/toast_01.png}

\caption{Toast table structure}
\label{fig:TOAST01} 
\end{center}

\end{figure}

The toast table is composed by three fields. The chunk\_id is an OID used to store the chunk identifiers. 
The chunk\_seq is an integer which stores the chunk orders. The chunk\_data is a bytea field containing the 
the actual data converted in a binary string.\newline 

The chunk size is normally 2k and is controlled at compile time by the symbol TOAST\_MAX\_CHUNK\_SIZE. The 
TOAST code is triggered by the value TOAST\_TUPLE\_THRESHOLD, also 2k by default. When the tuple's size is 
bigger than TOAST\_TUPLE\_THRESHOLD then the TOAST routines are triggered.\newline

The TOAST\_TUPLE\_TARGET, default 2 kB, governs the compression's behaviour. PostgreSQL will compress the 
datum to achieve a final size lesser than TOAST\_TUPLE\_TARGET. Otherwise the out of line storage is used.

TOAST offers four different storage strategies. Each strategy can be changed per column using the  ALTER 
TABLE SET STORAGE statement.
\begin{itemize}

\index{TOAST, storage strategies}
\item  PLAIN prevents either compression or out-of-line storage; It's the only storage available 
for fixed length data types.

\item  EXTENDED allows both compression and out-of-line storage. It is the default for most 
TOAST-able data types. Compression will be attempted first, then out-of-line storage if the row is 
still too big.

\item  EXTERNAL allows out-of-line storage but not compression. 

\item  MAIN allows compression but not out-of-line storage. Actually the out-of-line storage is 
still performed as last resort.

\end{itemize}

The out of line storage\index{TOAST, out of line storage} have the advantage of leaving out the 
stored data from the row versioning; if the TOAST data is not affected by the update there will be 
no dead row for the TOAST data. That's possible because the varlena is a mere pointer to the chunks 
and a new row version will affect only the pointer leaving the TOAST data unchanged.\newline
The TOAST table are stored like all the other relation's in the pg\_class table, the associated 
table can be found using a self join on the field reltoastrelid.\newline


\section{Tablespaces}\index{tablespaces,physical}
\label{sub:TBS-PHYSICAL}
PostgreSQL implements the tablespaces with the symbolic links. Inside the directory \$PGDATA/pg\_tblspc 
there are the links to the physical location. Each link is named after the tablespace's OID. Therefore the 
tablespaces are available only on the systems with the symbolic link support.\newline

Before the version 8.4 the tablespace symbolic link pointed directly to the referenced directory. This was 
a race condition when upgrading in place because the the location could clash with the upgraded cluster. 
From the version 9.0, the tablespace creates a sub directory directory in the tablespace location which 
is after the major version and the system catalogue version number. 
\newline

\begin{verbatim}

postgres@tardis:~$ ls -l /var/lib/postgresql/pg_tbs/ts_test
total 0
drwx------ 2 postgres postgres 6 Jun  9 13:01 PG_9.3_201306121

\end{verbatim}

The sub directory's name is a combination of the capital letters PG followed by the major version, 
truncated to the first two numbers, and the catalogue version number stored in the control file.\newline



\begin{verbatim}
postgres@tardis:~$ export PGDATA=/var/lib/postgresql/9.3/main
postgres@tardis:~$ /usr/lib/postgresql/9.3/bin/pg_controldata 
pg_control version number:            937
Catalog version number:               201306121
Database system identifier:           5992975355079285751
Database cluster state:               in production
pg_control last modified:             Mon 09 Jun 2014 13:05:14 UTC
.
.
.
WAL block size:                       8192
Bytes per WAL segment:                16777216
Maximum length of identifiers:        64
Maximum columns in an index:          32
Maximum size of a TOAST chunk:        1996
Date/time type storage:               64-bit integers
Float4 argument passing:              by value
Float8 argument passing:              by value
Data page checksum version:           0



PG_{MAJOR_VERSION\}_{CATALOGUE_VERSION_NUMBER}

\end{verbatim}



Inside the container directory the data files are organised in the same way as in base directory.
\ref{sec:PGDATA}.\newline

Moving a tablespace to another physical location it's not complicated but the cluster needs to be shut down.
With the cluster stopped the container directory can be safely copied to the new location. The receiving 
directory must have the same permissions  like the origin's. The symbolic link must be recreated to point 
to the new physical location. At the cluster's start the change will be automatically resolved from 
the symbolic link.\newline

Until PostgreSQL 9.1 the tablespace location was stored into the field spclocation in the system table 
pg\_tablespace\index{pg\_tablespace}. From the version 9.2 the spclocation field is removed and the 
tablespace's location is resolved on the fly using the function 
pg\_tablespace\_location(tablespace\_oid).\newline

This function can be used to query the system catalogue about the tablespaces. In this simple example the 
query returns the tablespace's location resolved from the OID. 

\begin{lstlisting}[style=pgsql]
postgres=# 
                SELECT 
                        pg_tablespace_location(oid),
                        spcname 
                FROM 
                        pg_tablespace
                ;
        
       pg_tablespace_location       |  spcname   
------------------------------------+------------
                                    | pg_default
                                    | pg_global
 /var/lib/postgresql/pg_tbs/ts_test | ts_test
(3 rows)

\end{lstlisting}

Because the function pg\_tablespace\_location returns the empty string for the system tablespaces, a better 
approach is combining the CASE construct with the function current\_settings and build the absolute path 
for the system tablespaces.

\begin{lstlisting}[style=pgsql]
 postgres=# SELECT current_setting('data_directory');
       current_setting        
------------------------------
 /var/lib/postgresql/9.3/main
(1 row)

postgres=# 
SELECT 
        CASE
                WHEN 
                                pg_tablespace_location(oid)=''
                        AND     spcname='pg_default'
                THEN
                        current_setting('data_directory')||'/base/'
                WHEN 
                                pg_tablespace_location(oid)=''
                        AND     spcname='pg_global'
                THEN
                        current_setting('data_directory')||'/global/'
        ELSE
                pg_tablespace_location(oid)
        END
        AS      spclocation,
                
        spcname 
FROM 
        pg_tablespace;
             spclocation              |  spcname   
--------------------------------------+------------
 /var/lib/postgresql/9.3/main/base/   | pg_default
 /var/lib/postgresql/9.3/main/global/ | pg_global
 /var/lib/postgresql/pg_tbs/ts_test   | ts_test
(3 rows)

\end{lstlisting}

Another useful function the pg\_tablespace\_databases(tablespace\_oid) can help us to find the databases 
with the relations on a certain tablespace.\newline

The following example uses this function again with a CASE construct for building the database having 
objects on a specific tablespace, in our example the ts\_test created in \ref{sub:TBS-LOGICAL}.\newpage
\begin{lstlisting}[style=pgsql]
 db_test=# 
 SELECT
        datname,
        spcname,
        CASE
                WHEN 
                                pg_tablespace_location(tbsoid)=''
                        AND     spcname='pg_default'
                THEN
                        current_setting('data_directory')||'/base/'
                WHEN 
                                pg_tablespace_location(tbsoid)=''
                        AND     spcname='pg_global'
                THEN
                        current_setting('data_directory')||'/global/'
        ELSE
                pg_tablespace_location(tbsoid)
        END
        AS      spclocation
FROM
        pg_database dat,
        (
                SELECT
                        oid as tbsoid,
                        pg_tablespace_databases(oid) as datoid,
                        spcname 
                FROM 
                        pg_tablespace where spcname='ts_test'
        ) tbs
WHERE
        dat.oid=tbs.datoid
;
 datname | spcname |            spclocation             
---------+---------+------------------------------------
 db_test | ts_test | /var/lib/postgresql/pg_tbs/ts_test
(1 row)

\end{lstlisting}



\section{MVCC} \label{sec:MVCC}\index{MVCC} 
The multiversion concurrency control is used in PostgreSQL to implement the  transactional model seen in 
\ref{sec:TRANSACTION}.\newline

At logical level this is completely transparent to the user and the new row versions become visible 
after the commit, accordingly with the transaction isolation level. \newline

At physical level we have for each new row version, the insert's XID stored into the t\_xmin field which is 
used by the internal semantic to determine the row visibility. 

Because the XID is a 32 bit quantity, it wraps at 4 billions. When this happens theoretically all 
the tuples should suddenly disappear because they switch from in the current XID's past to its future in 
the well known XID wraparound failure,\index{XID wraparound failure}. In the old PostgreSQL versions this 
was a serious problem which forced the administrators to dump/reload the entire cluster into a freshly 
initialised new data area every 4 billion of transactions.\newline 

In PostgreSQL 7.2 was introduced a new comparison method for the XID, the 
\begin{math}modulo-2^{32}\end{math} arithmetic. It was also introduced a special XID, the 
FrozenXID\footnote{The FrozenXID's value is 2. The docs of PostgreSQL 
7.2 also mention the BootstrapXID which value is 1} assumed as always in the past. With the new 
comparison method, for any arbitrary XID exists 2 billion of transactions in the future and 2 billion 
transactions in the past.\newline

When the age of the tuple's t\_xmin becomes old the periodic VACUUM\index{VACUUM} freezes the ageing tuple 
changing its t\_xmin to the FrozenXID always in the past. In the pg\_class and the pg\_database tables 
there are  two dedicated fields to track the age of the oldest XID. The value stored in those tables 
have little meaning if not processed through the function age() which shows the number of transactions 
between the current XID and the value stored in the system catalogue. \newline

This following query returns all the databases, the corresponding datfrozenxid and the XID's age.\newpage

\begin{lstlisting}[style=pgsql]
 postgres=# 
        SELECT 
                datname,
                age(datfrozenxid),
                datfrozenxid 
        FROM 
                pg_database;
    datname    | age  | datfrozenxid 
---------------+------+--------------
 template1     | 4211 |          679
 template0     | 4211 |          679
 postgres      | 4211 |          679
 db_test       | 4211 |          679

\end{lstlisting}

When a tuple's age is more than 2 billions the tuple simply disappears  from the cluster. Before the 
version 8.0 there was no alert or protection against the XID wraparound failure. Since then it was 
introduced a passive mechanism which emits messages in the activity log when the age of datfrozenxid 
is less than ten million transactions from the wraparound point.

A message like this is quite serious and should not be ignored.
\begin{smallverbatim}
WARNING:  database "test_db" must be vacuumed within 152405486 transactions
HINT:  To avoid a database shutdown, execute a database-wide VACUUM in 
"test_db".
\end{smallverbatim}

The autovacuum daemon in this case acts like a watchdog and starts vacuuming the tables with ageing 
tuples even  if autovacuum is turned off in the cluster. There is another protection, quite radical, 
if for some reasons one of the database's datfrozenxid is at one million transactions from the 
wraparound point. In this case the cluster shuts down and refuse to start again. The only option in 
this case is to run the postgres process in single-user backend and execute the VACUUM on the 
affected relations.\newline

The debian package's configuration is quite odd, putting the configuration files in the /etc/postgresql 
instead of the data area. The following example is the standalone backend's call for the debian's packaged 
default cluster main.

\begin{verbatim}

postgres@tardis:~/tempdata$ /usr/lib/postgresql/9.3/bin/postgres \
--single -D /var/lib/postgresql/9.3/main/base/ \
--config-file=/etc/postgresql/9.3/main/postgresql.conf

PostgreSQL stand-alone backend 9.3.5
backend> 

\end{verbatim}

The database interface in single user mode and does not have all the sophisticated features 
like the client psql. Anyway with a little knowledge of SQL it's possible to find the database(s) 
causing the shutdown and fix it.
\index{postgres, single user mode}\index{XID wraparound failure, fix}

\begin{verbatim}
backend> SELECT datname,age(datfrozenxid) FROM pg_database ORDER BY 2 DESC;

1: datname     (typeid = 19, len = 64, typmod = -1, byval = f)
2: age (typeid = 23, len = 4, typmod = -1, byval = t)
----
1: datname = "template1" (typeid = 19, len = 64, typmod = -1, byval = f)
2: age = "2146435072"  (typeid = 23, len = 4, typmod = -1, byval = t)
----
1: datname = "template0" (typeid = 19, len = 64, typmod = -1, byval = f)
2: age = "10"  (typeid = 23, len = 4, typmod = -1, byval = t)
----
1: datname = "postgres"  (typeid = 19, len = 64, typmod = -1, byval = f)
2: age = "10"  (typeid = 23, len = 4, typmod = -1, byval = t)
----

\end{verbatim}

The age function shows how old is the last XID not yet frozen. In our example the template1
database have an age of 2146435072, one million transactions to the wraparound. We can then exit 
the backend with CTRL+D and restart it again in the in single user mode specifying the database 
name. A VACUUM will get rid of the problematic xid.

\begin{verbatim}
postgres@tardis:~/tempdata$ /usr/lib/postgresql/9.3/bin/postgres \
--single -D /var/lib/postgresql/9.3/main/base/ \
--config-file=/etc/postgresql/9.3/main/postgresql.conf \
template1
                                
backend> SELECT current_database();
1: current_database (typeid = 19, len = 64, typmod = -1, byval = f)
----
1: current_database = "template1" (typeid = 19, len = 64, typmod = -1, byval = f)
----

backend> VACUUM FREEZE;
\end{verbatim}

This procedure must be repeated for any database with very old XID.\newline

Because the new rows generation at update time, this can lead to an unnecessary table and index bloat.
PostgreSQL with the Heap Only Tuples (HOT)\index{HOT strategy} strategy can limit the unavoidable bloat 
caused by the updates. HOT's main goal is to keep the new row versions into the same page. 

The MVCC is something to consider at design time. Ignoring the way PostgreSQL manages the physical tuples 
can result in data bloat and lead in general to poor performances.

\chapter{Maintenance}
\label{cha:MAINTENANCE}\index{Maintenance}
The database maintenance is something crucial for the efficiency of the data access, the integrity 
and the reliability. Any database sooner or later will need a proper maintenance plan. \newline

When a new tuple's version is generated by an update it can be put everywhere there's free space. 
Frequent updates can result in tuples moving across the data pages many and many times leaving a 
trail of dead tuples behind them. Because the dead tuples are physically stored but no 
longer visible this creates an extra overhead causing the table to bloat.
Indices makes things more complicated because when a tuple changes page the index entry is updated 
to point the new page and because of the index's ordered structure, the bloating is more 
probable than the table. 


\section{vacuum}\index{VACUUM}
\label{sec:VACUUM}
VACUUM is a PostgreSQL specific command which reclaims back the dead tuple's space. When called 
without specifying a target table, the command processes all the tables in the database. Running 
regulary VACUUM have some beneficial effects.

\begin{itemize}
 \item It reclaims back the dead tuple's disk space.
 \item It updates the visibility map making the index scans run faster.
 \item It freezes the tuples with old XID protecting from the XID wraparound\index{XID wraparound} 
data loss
\end{itemize}

The optional ANALYZE clause also gather the statistics on processed table, more details here 
\ref{sec:ANALYZE}.\newline

A standard VACUUM's run, frees the space used by the dead rows inside the data files but doesn't 
returns the space to the operating system. VACUUM doesn't affects the common database activity 
but prevents any schema change on the processed table. Because the pages are rewritten, a VACUUM 
run increases substantially the I/O activity. \newline

The presence of one or more empty pages in the table's end can be removed by VACUUM if an 
exclusive lock on the relation can be obtained immediately. When this happens the table is scanned 
backward to find all the empty pages and then it's truncated to the first not empty page. The index 
pages are scanned as well and the dead tuples are also cleared. The VACUUM's truncate scan works 
only on the heap data files. VACUUM'S performances are influenced by the maintenance\_work\_mem 
only if the table have indices, otherwise the VACUUM will run the cleanup sequentially without 
storing the tuple's references for the index cleanup.\newline

To show the effect of the maintenance\_work\_mem  let's build build a simple table with 10 
million rows. 


\begin{lstlisting}[style=pgsql]
postgres=# CREATE TABLE t_vacuum 
        (
                i_id serial,
                t_ts_value timestamp with time zone DEFAULT clock_timestamp(),
                t_value text,
                CONSTRAINT pk_t_vacuum PRIMARY KEY  (i_id)
        )
;
CREATE TABLE

postgres=# INSERT INTO t_vacuum
        (t_value)
SELECT 
         md5(i_cnt::text)
FROM
(
        SELECT
                generate_series(1,10000000) as i_cnt
) t_cnt
;
INSERT 0 10000000


\end{lstlisting}
To have a statical environment we'll disable the table's autovacuum. More infos on autovacuum here 
\ref{sec:AUTOVACUUM}.
We'll also increase the session's verbosity to look out what's happening during the VACUUM's 
 run.\newline
\begin{lstlisting}[style=pgsql]
postgres=# ALTER TABLE t_vacuum 
        SET 
                (
                        autovacuum_enabled = false, 
                        toast.autovacuum_enabled = false
                )
;
ALTER TABLE


SET client_min_messages='debug';

\end{lstlisting}

We are now executing a complete table rewrite running an UPDATE without the WHERE condition. 
This will create 10 millions of dead rows.\newline

\begin{lstlisting}[style=pgsql]
postgres=# UPDATE t_vacuum 
        SET 
                t_value = md5(clock_timestamp()::text)
;
UPDATE 10000000

\end{lstlisting}

Before running the VACUUM we'll change the maintenance\_work\_mem to a small value enabling the the 
timing to check the query duration.\newline

\begin{lstlisting}[style=pgsql]
postgres=# SET maintenance_work_mem ='20MB';
SET
postgres=# \timing
Timing is on.

postgres=# VACUUM t_vacuum;
DEBUG:  vacuuming "public.t_vacuum"
DEBUG:  scanned index "pk_t_vacuum" to remove 3495007 row versions
DETAIL:  CPU 0.80s/4.56u sec elapsed 21.36 sec.
DEBUG:  "t_vacuum": removed 3495007 row versions in 36031 pages
DETAIL:  CPU 0.63s/0.56u sec elapsed 19.31 sec.
DEBUG:  scanned index "pk_t_vacuum" to remove 3495007 row versions
DETAIL:  CPU 0.67s/4.18u sec elapsed 15.28 sec.
DEBUG:  "t_vacuum": removed 3495007 row versions in 36031 pages
DETAIL:  CPU 0.67s/0.53u sec elapsed 18.07 sec.
DEBUG:  scanned index "pk_t_vacuum" to remove 3009986 row versions
DETAIL:  CPU 0.53s/2.86u sec elapsed 12.29 sec.
DEBUG:  "t_vacuum": removed 3009986 row versions in 31031 pages
DETAIL:  CPU 0.47s/0.52u sec elapsed 20.06 sec.
DEBUG:  index "pk_t_vacuum" now contains 10000000 row versions in 82352 pages
DETAIL:  10000000 index row versions were removed.
0 index pages have been deleted, 0 are currently reusable.
CPU 0.00s/0.00u sec elapsed 0.00 sec.
DEBUG:  "t_vacuum": found 10000000 removable, 10000000 nonremovable row versions in 206186 out of 
206186 pages
DETAIL:  0 dead row versions cannot be removed yet.
There were 0 unused item pointers.
0 pages are entirely empty.
CPU 5.92s/17.08u sec elapsed 154.10 sec.
DEBUG:  vacuuming "pg_toast.pg_toast_28499"
DEBUG:  index "pg_toast_28499_index" now contains 0 row versions in 1 pages
DETAIL:  0 index row versions were removed.
0 index pages have been deleted, 0 are currently reusable.
CPU 0.00s/0.00u sec elapsed 0.00 sec.
DEBUG:  "pg_toast_28499": found 0 removable, 0 nonremovable row versions in 0 out of 0 pages
DETAIL:  0 dead row versions cannot be removed yet.
There were 0 unused item pointers.
0 pages are entirely empty.
CPU 0.00s/0.00u sec elapsed 0.00 sec.
VACUUM
Time: 154143.383 ms
postgres=# 


\end{lstlisting}

During the VACUUM the the maintenance\_work\_mem is used to store an array of TCID referencing 
the dead tuples for the index cleanup. If the maintenance\_work\_mem is small and the dead rows are 
many, the memory fills up often. When this happens the table scan pauses and the index is 
scanned searching for the tuples stored into the array. When the index scan is complete the array of 
TCID is emptied and the table's scan resumes. Increasing the maintenance\_work\_mem to 2 
GB\footnote{In order to have the table in the same conditions, a VACUUM FULL and a new update has 
been runt before the conventional VACUUM.} the index scan is executed in one single run 
resulting in a VACUUM 32 seconds faster.\newline

\begin{lstlisting}[style=pgsql]
postgres=# SET maintenance_work_mem ='2GB';
SET

postgres=# VACUUM t_vacuum;
DEBUG:  vacuuming "public.t_vacuum"
DEBUG:  scanned index "pk_t_vacuum" to remove 10000000 row versions
DETAIL:  CPU 1.58s/8.45u sec elapsed 52.41 sec.
DEBUG:  "t_vacuum": removed 10000000 row versions in 103093 pages
DETAIL:  CPU 1.78s/1.41u sec elapsed 33.90 sec.
DEBUG:  index "pk_t_vacuum" now contains 10000000 row versions in 82352 pages
DETAIL:  10000000 index row versions were removed.
0 index pages have been deleted, 0 are currently reusable.
CPU 0.00s/0.00u sec elapsed 0.00 sec.
DEBUG:  "t_vacuum": found 10000000 removable, 10000000 nonremovable row versions in 206186 out of 
206186 pages
DETAIL:  0 dead row versions cannot be removed yet.
There were 0 unused item pointers.
0 pages are entirely empty.
CPU 5.62s/13.64u sec elapsed 121.99 sec.
DEBUG:  vacuuming "pg_toast.pg_toast_28499"
DEBUG:  index "pg_toast_28499_index" now contains 0 row versions in 1 pages
DETAIL:  0 index row versions were removed.
0 index pages have been deleted, 0 are currently reusable.
CPU 0.00s/0.00u sec elapsed 0.00 sec.
DEBUG:  "pg_toast_28499": found 0 removable, 0 nonremovable row versions in 0 out of 0 pages
DETAIL:  0 dead row versions cannot be removed yet.
There were 0 unused item pointers.
0 pages are entirely empty.
CPU 0.00s/0.00u sec elapsed 0.00 sec.
VACUUM
Time: 122021.251 ms


\end{lstlisting}

A table without indices does use the maintenance\_work\_mem. For example if we run 
the VACUUM after dropping the table's primary key the execution is faster even with the low 
maintenance\_work\_mem setting.\newline

\begin{lstlisting}[style=pgsql]

postgres=# SET maintenance_work_mem ='20MB';
SET
postgres=# \timing
Timing is on.

postgres=# ALTER TABLE t_vacuum DROP CONSTRAINT pk_t_vacuum;
DEBUG:  drop auto-cascades to index pk_t_vacuum
ALTER TABLE
Time: 182.737 ms

postgres=# VACUUM t_vacuum;
DEBUG:  vacuuming "public.t_vacuum"
DEBUG:  "t_vacuum": removed 10000000 row versions in 103093 pages
DEBUG:  "t_vacuum": found 10000000 removable, 10000000 nonremovable row versions in 206186 out of 
206186 pages
DETAIL:  0 dead row versions cannot be removed yet.
There were 0 unused item pointers.
0 pages are entirely empty.
CPU 2.16s/4.53u sec elapsed 47.30 sec.
DEBUG:  vacuuming "pg_toast.pg_toast_28499"
DEBUG:  index "pg_toast_28499_index" now contains 0 row versions in 1 pages
DETAIL:  0 index row versions were removed.
0 index pages have been deleted, 0 are currently reusable.
CPU 0.00s/0.00u sec elapsed 0.00 sec.
DEBUG:  "pg_toast_28499": found 0 removable, 0 nonremovable row versions in 0 out of 0 pages
DETAIL:  0 dead row versions cannot be removed yet.
There were 0 unused item pointers.
0 pages are entirely empty.
CPU 0.00s/0.00u sec elapsed 0.00 sec.
VACUUM
Time: 48823.132 ms




\end{lstlisting}

The table seen in the example begins with a size of 806 MB . After the update the table double its 
size which remains the same during the VACUUM runs the updates. This happens because after the 
first insert the table had all the rows packed together; the update added in the table's bottom the 
new row versions leaving the previous 10 millions row on the table's top as dead tuples. The 
VACUUM's run cleared the space on the table's top but weren't able to truncate because all 
the rows packed in the table's bottom. Running a new UPDATE followed by VACUUM would free the space 
in the table's bottom and a truncate scan would succeed but only if there's no tuple in the table's 
end free space. To check if the vacuum is running effectively the tables should show an initial 
growt followed by a substantial size stability in time. This happens only if the new rows are 
versioned at the same rate of the old rows clear down.\newline

The XID wraparound failure protection is performed automatically by VACUUM which when it finds a 
live tuple with a t\_xmin's age bigger than the GUC parameter vacuum\_freeze\_min\_age, then it 
replaces the tuple's creation XID with the FrozenXID preserving the tuple's visibility forever. 
Because VACUUM by default skips the pages without dead tuples it will miss some aging tuples. 
That's the reason why it's present a second GUC parameter, vacuum\_freeze\_table\_age, which 
triggers a VACUUM's full table scan when the table's relfrozenxid age exceeds the value.\newline

VACUUM accepts the FREEZE \index{VACUUM FREEZE} clause which forces a complete tuple freeze 
regardless to the age. That's equivalent to run the VACUUM setting the vacuum\_freeze\_min\_age to 
zero.


\section{analyze}
\label{sec:ANALYZE}


\section{reindex}
\begin{comment}
  . A B-tree index is a tree of 
table's implementation, An index is The tuples moving across the data files is the index entries 
shall be updated. A B-tree index entry basically carries the indexed value with the pointer to the 
page containing the tuple. When the tupe change page the index entry shall be updated. This require 
\end{comment}

\section{VACUUM FULL and CLUSTER}


\section{The autovacuum daemon}
\label{sec:AUTOVACUUM}
\chapter{Backup}
\label{cha:BACKUP}
The hardware is subject to faults. In particular if the storage is lost the entire data 
infrastructure becomes inaccessible, sometime for good. Also human errors, like wrong delete or table drop 
can happen. A solid backup strategy is the best protection against these problems and much more. The 
chapter covers the logical backup with pg\_dump. 

\section{pg\_dump at glance}
\label{sec:PGDUMP}
As seen in \ref{sub:PGDUMP}, pg\_dump\index{pg\_dump} is the PostgreSQL's utility for saving 
consistent snapshots of the databases. The usage is quite simple and if launched without options it tries 
to connect to the local cluster with the current user redirecting the dump's output to the standard 
output.\newline

The help gives many useful information.

\begin{verbatim}
postgres@tardis:~/dump pg_dump --help
pg_dump dumps a database as a text file or to other formats.

Usage:
  pg_dump [OPTION]... [DBNAME]

General options:
  -f, --file=FILENAME          output file or directory name
  -F, --format=c|d|t|p         output file format (custom, directory, tar,
                               plain text (default))
  -j, --jobs=NUM               use this many parallel jobs to dump
  -v, --verbose                verbose mode
  -V, --version                output version information, then exit
  -Z, --compress=0-9           compression level for compressed formats
  --lock-wait-timeout=TIMEOUT  fail after waiting TIMEOUT for a table lock
  -?, --help                   show this help, then exit

Options controlling the output content:
  -a, --data-only              dump only the data, not the schema
  -b, --blobs                  include large objects in dump
  -c, --clean                  clean (drop) database objects before recreating
  -C, --create                 include commands to create database in dump
  -E, --encoding=ENCODING      dump the data in encoding ENCODING
  -n, --schema=SCHEMA          dump the named schema(s) only
  -N, --exclude-schema=SCHEMA  do NOT dump the named schema(s)
  -o, --oids                   include OIDs in dump
  -O, --no-owner               skip restoration of object ownership in
                               plain-text format
  -s, --schema-only            dump only the schema, no data
  -S, --superuser=NAME         superuser user name to use in plain-text format
  -t, --table=TABLE            dump the named table(s) only
  -T, --exclude-table=TABLE    do NOT dump the named table(s)
  -x, --no-privileges          do not dump privileges (grant/revoke)
  --binary-upgrade             for use by upgrade utilities only
  --column-inserts             dump data as INSERT commands with column names
  --disable-dollar-quoting     disable dollar quoting, use SQL standard quoting
  --disable-triggers           disable triggers during data-only restore
  --exclude-table-data=TABLE   do NOT dump data for the named table(s)
  --inserts                    dump data as INSERT commands, rather than COPY
  --no-security-labels         do not dump security label assignments
  --no-synchronized-snapshots  do not use synchronized snapshots in parallel jobs
  --no-tablespaces             do not dump tablespace assignments
  --no-unlogged-table-data     do not dump unlogged table data
  --quote-all-identifiers      quote all identifiers, even if not key words
  --section=SECTION            dump named section (pre-data, data, or post-data)
  --serializable-deferrable    wait until the dump can run without anomalies
  --use-set-session-authorization
                               use SET SESSION AUTHORIZATION commands instead of
                               ALTER OWNER commands to set ownership

Connection options:
  -d, --dbname=DBNAME      database to dump
  -h, --host=HOSTNAME      database server host or socket directory
  -p, --port=PORT          database server port number
  -U, --username=NAME      connect as specified database user
  -w, --no-password        never prompt for password
  -W, --password           force password prompt (should happen automatically)
  --role=ROLENAME          do SET ROLE before dump

\end{verbatim}

\subsection{Connection options}
\index{pg\_dump, connection options}
The connection options are used to specify the way the program connects to the cluster. All the options are 
straightforward except for the password. Usually the PostgreSQL clients don't accept the plain password as 
parameter. However is still possible to connect without specifying the password using the 
environmental variable PGPASSWORD\index{PGPASSWORD} or using the password file.\newline

Using the variable PGPASSWORD is considered not secure and shouldn't be used if  not trusted users are 
accessing the server. The password file is a text file named saved in the users's home directory as 
.pgpass . The file must be readable only by the user, otherwise the client will refuse to read it.\newline
Each line specifies a connection in a fixed format.
\index{password file}
\begin{verbatim}
hostname:port:database:username:password
\end{verbatim}

The following example specifies the password for the connection to the host tardis, port 5432, database 
db\_test and user usr\_test.

\begin{verbatim}
tardis:5432:db_test:usr_test:testpwd
\end{verbatim}

\subsection{General options}
\index{pg\_dump, general options}
The general options are used to control the backup's output and format. 

The switch -f sends the backup on the specified FILENAME.\newline

The switch \index{pg\_dump, output formats} -F specifies the backup format and requires a second 
option to tell pg\_dump which format to use. The allowed formats are \textit{c d t p} respectively 
\textit{custom directory tar plain}.\newline

If the parameter is omitted pg\_dump uses the plain text format. not compressed and suitable for the direct 
load using psql. \newline

The the custom and the directory format are the most versatile backup formats. They give compression and 
flexibility at restore time. Both have the parallel and selective restore option.\newline

The directory format stores the schema dump in a toc file. Each table's content is then saved in a 
compressed file inside the target directory specified with the -f switch. From the version 9.3 this format 
allows the parallel dump functionality\index{pg\_dump, parallel export}. \newline

The tar format stores the dump in the well known tape archive format. This format is compatible 
with the directory format, does not compress the data and there is the limit of 8 GB for the individual
table.\newline

The -j option specifies the number of jobs to run in parallel when dumping the data. This feature is 
available from the version 9.3 and uses the transaction's snapshot export to have a consistent dump over 
the multiple export jobs. The switch is usable only with the directory format and only with PostgreSQL 9.2 
and later.\newline 

The option -Z specifies the compression level for the compressed formats. The default is 5 
resulting in a dumped archive from 5 to 8 times smaller than the original database. 

The option --lock-wait-timeout is the number of milliseconds for the table's lock acquisition. 
When expired the dump will fail. Is useful to avoid the program to wait forever for a table lock 
but can result in failed backups if value is too much low.

\subsection{Output options}
\index{pg\_dump, output options}
\label{sub:PGDUMPOUTPUT}
The output options control backup output. Some of those options are meaningful only under certain 
conditions, some others are quite obvious.\newline

The -a option sets the data only export. Separating schema and data have some effects at restore 
time, in particular with the performance. We'll see in the detail in \ref{cha:RESTORE} how to 
build an efficient two phase restore.\newline

The -b option exports the large objects. This is the default setting except if the -n switch is 
used. In this case the -b is required to export the large objects.\newline

The options -c and -C are meaningful only for the plain output format. They respectively add the 
DROP and CREATE command before the object's DDL. For the archive formats the same option exists for 
pg\_restore.\newline

The -E specifies the character encoding for the archive. If not set the database's encoding is 
used.\newline 

The -n switch is used to dump the named schema only. It's possible to specify multiple -n switches 
to select many schema or using the wildcards. However despite the efforts of pg\_dump to get all 
the dependencies resolved, something could be missing. There's no guarantee the resulting archive 
can be successfully restored.\newline

The -N switch does the opposite of the -n switch. Excludes the named database schema from the backup. The 
switch accepts wildcards and it's possible to specify multiple schema with multiple -N switches. When both 
-n and -N are given, the behaviour is to dump just the schema that match at least one -n switch but no -N 
switches. \newline

The -o switch option dumps the object id as part of the table for every table. This options should be 
used only if the OIDs are part of the design. \newline

The -O switch have effects only on plain text exports and does not dump statements setting object 
ownership.\newline

The -s switch option dumps only the database schema.\newline

The -S switch is meaningful only for plain text exports. The switch specifies the super user for disabling 
and enabling the triggers if the export is performed with the option --disable-triggers. \newline

The -t switch is used to dump the named table only. It's possible to specify multiple tables using 
the wildcards or specifying the -t many times.\newline

The -T skips the named table in the dump. It's possible to exclude multiple tables using 
the wildcards or specifying the -T many times.\newline

The switch -x does not save the grant/revoke commands for setting the privileges.\newline

The switch --binary-upgrade is used only for the in place upgrade program pg\_upgrade. Is not intended for 
general usage. 

The switch --insert option dumps the data as INSERT command instead of the COPY. The restore with this 
option is very slow because each statement is parsed and executed individually.\newline

The switch --column-inserts results in the data exported as INSERT commands with all the column 
names specified.\newline

The switch --disable-dollar-quoting disables the dollar quoting for the function's body and uses the 
standard SQL quoting.\newline

The switch --disable-triggers save the statements for disabling  the triggers before the data load and the 
enabling them back after the data load. Disabling the triggers will ensure the foreign keys will not cause 
errors during the data load. This switch have effect only for the plain text export.\newline

The switch --exclude-table-data=TABLE skips the data dump for the named table. The same rules of the -t and 
-T apply to this switch.\newline


The switch --no-security-labels doesn't include the security labels into the dump file.\newline

The switch --no-synchronized-snapshots  is used to run a parallel export with the pre 9.2 
databases. Because the snapshot export feature is missing this means the database shall not change 
state until all the exporting jobs are connected.\newline

The switch --no-tablespaces skips the tablespace assignments.\newline

The switch  --no-unlogged-table-data does not export data for the unlogged relations.\newline

The switch  --quote-all-identifiers  cause all the identifiers to be enclosed in double quotes. \newline

The switch --section option specifies one of the three export's sections. The first section is the 
pre-data, which saves the definitions for the tables, the views and the functions. The second section is the 
data which saves the table's contents. The third section is the post-data which saves the constraints, the 
indices and the eventual GRANT REVOKE commands . This switch applies only to the plain format. \newline

The switch --serializable-deferrable uses a serializable transaction for the dump, to ensure the database 
state is consistent. The dump execution waits for a point in the transaction stream without 
anomalies to avoid the risk of serialization\_failure. The option is not useful for the backup used 
only for disaster recovery and should be used only when the dump should reload into a read only database 
which needs to get a consistent state compatible with the origin's database.\newline

The switch --use-set-session-authorization sets the objects ownership using the command SET SESSION 
AUTHORIZATION instead of the ALTER OWNER. SET SESSION AUTHORIZATION requires the super user privileges 
whereas ALTER OWNER doesn't.


\section{Performance tips}
\index{pg\_dump, performance tips}
Despite the fact pg\_dump doesn't affects the running queries, its strict transactional approach 
have some effects on the affected schema. Any alter schema is blocked until the backup's end. The 
vacuum efficiency is affected as well, because all the dead rows generated during the backup's run, 
cannot be reclaimed being potentially required by the backup's running transaction.\newline

There are some tips to improve the backup's speed.

\subsection{Avoid remote backups}
The pg\_dump can connect to remote databases same like any other PostgreSQL client.
It seems reasonable then to use the program installed on a centralised storage and to dump locally 
from the remote cluster.\newline 
Unfortunately even using the compressed format, the entire database flows uncompressed and in 
clear, from the server to the remote pg\_dump. The compression happens locally when the data is 
received.\newline
This approach expose also a network security issue. If the environment is not trusted then the 
remote connection must happen on a secure channel. This add an extra overhead to the transfer and 
any failure on this layer will fail the entire backup.\newline
A far better approach is to save locally the database, using the local connection if possible, and 
then copy the entire dump file using a secure transfer protocol like scp or sshfs.

\subsection{Skip replicated tables}
If the database is configured as logical slave in slony or londiste for example, backing up the 
replicated table's data is not important as the contents are re synchronised from the master when 
the node is attached to the replication system. The switch --exclude-table-data=TABLE is then 
useful for dumping the table's definition only without the contents.

\subsection{Check for slow cpu cores}
PostgreSQL is a multitasking but not a multithreaded database system. Each backend is attached 
to just one cpu. The pg\_dump opens one backend connection to the cluster in order to 
export the database objects. The pg\_dump process receives the data flow from the backend and it 
saves performing also the optional compression. In this scenario the cpu power is critical in order 
to avoid a bottleneck. This could be helped using the parallel export offered by pg\_dump from the 
version 9.3. The functionality is implemented via  the snapshot exports. As this was introduced 
with PostgreSQL 9.2 the paralle export can happen only from this version and only if the output 
format is directory . 

\subsection{Check for the available locks}
PostgreSQL at various level uses the locks to ensure the data consistency at various levels. For 
example, when a table is read an access share lock on the relation is put in order to avoid any 
structure change. Any backend issuing an ALTER TABLE which affect the table structure, will wait 
for the lock to be released before acquiring itself an exclusive lock and then perform the change.
The relation's locks are stored into the pg\_locks table. This table is quite unique because have 
a limited number of rows. The maximum number of table's lock slot is determine with this simple 
formula.\newline
\begin{math}
 max_locks_per_transaction * (max_connections + max_prepared_transactions)
\end{math}\newline

The default configuration permits have only 6400 table's lock slots. This value is generally OK. 
However, if the database have a great number of relations, a full backup, pg\_dump could hit the 
slot limit and fail with an out of memory error.\newline
All the three GUC parameters require a restart to apply the new changes so is very important to 
plan the change before the limit is reached. 

\subsection{Try to use flexible formats}
This is more a good practice suggestion rather a performance tip. Exporting the database in plain 
text have some advantages. Is possible to load the dump just using psql and any file's corruption 
can be managed in a simple way; if the damage is limited of course. The custom and directory 
formats need the pg\_restore utility for the restoration. We'll take a look to this approach in 
\ref{cha:RESTORE}. Anyway, the custom and directory formats have alongside with the compression, 
the parallel restore feature and the selective restoration. The compression can be fine tuned to 
suit the export's nature. In this era of ``big data'' is something to consider seriously.


\section{pg\_dump under the bonnet}
\label{sec:PGDUMPINT}
\index{pg\_dump, internals}
The pg\_dump source code gives a very good picture of what exactly the backup 
software does. The process runs into with fixed transaction's isolation level accordingly with the 
server's version. 
The distinction is required because, becoming PostgreSQL more sophisticated at each major release, 
the isolation levels became more and more strict with their meanings.\newline
More informations about the transactions are in \ref{sec:TRANSACTION}.\newline
From PostgreSQL 9.1 the transaction serializable became a real serialisation. The 
transaction's behaviour offered by the serializable in the version up to 9.0 were assigned to the 
REPETABLE READ transaction's isolation, its real kind. The SERIALIZABLE transaction's isolation 
level is still used with DEFERRABLE option when pg\_dump is executed with 
the option --serializable-deferrable as seen in \ref{sub:PGDUMPOUTPUT}. The switch have effect only 
on the remote server with version 9.1 and later. The transaction is also set to READ ONLY, when 
supported by the server, in order to limit the XID generation. \newline

\begin{table}[H]
  \begin{tabular}{ll}
    Server version & Command    \\ 
    \hline
    \textgreater=  9.1 &  REPEATABLE READ, READ ONLY        \\
    \textgreater= 9.1 with --serializable-deferrable  &  SERIALIZABLE, READ ONLY, DEFERRABLE  \\
    \textgreater= 7.4 &  SERIALIZABLE READ ONLY   \\
    \textless 7.4 &  SERIALIZABLE \\
  \end{tabular}
  \caption{\label{tab:TRNPGDUMP}pg\_dump's transaction isolation levels }
\end{table}

From the version 9.3 pg\_dump supports the parallel dump using the feature seen in  
\ref{sub:SNAPEXPORT}. As the snapshot export is supported from the version 9.2 this permit a 
parallel dump from the older major version if using the newer backup program. However, 
pg\_dump accepts the option --no-synchronized-snapshots in order to dump in parallel jobs 
from the older versions. The data is not supposed to be consistent if there are  read write 
transactions during this kind of export. To have a consistent export in this case all the 
transactions which modify the data must be stopped meanwhile the export is in progress.\newline 

When exporting in parallel the only permitted format is the directory. The pg\_restore program 
since the version 9.3 supports also the directory format for the paralelle data restoration.
The combination of the parallel jobs backup and the parallel jobs restore, can improve massively 
either the backup and the recovery speed in case of disaster.

\section{pg\_dumpall}
\index{pg\_dumpall}
The pg\_dumpall program is more very basic version of pg\_dump. It's usage is mainly for 
dumping all the databases in the cluster. Having lesser options than pg\_dump is also less 
flexible. However one option is absolutely useful and should be included in any backup plan to 
ensure a rapid disaster recovery.\newline

The\index{pg\_dumpall, global objects} --globals-only option saves on the standard output all the 
cluster wide options like the tablespaces, the roles and the privileges. The passwords are dumped as 
well, encrypted in md5 format if stored this way. The utility output is just plain text. 
The best way to save the globals is to specify the -f option followed by the file name. This file 
can be loaded into an empty cluster to recreate the global objects. The tablespaces, if any present, 
must have the filesystem already in place before running the sql as PostgreSQL doesn't create the 
filesystem structure.\newline

This example shows the program call and the contents of the output file.
\begin{lstlisting}[style=pgsql]
postgres@tardis:~/dmp$ pg_dumpall --globals-only -f main_globals.sql
postgres@tardis:~/dmp$ cat main_globals.sql 
--
-- PostgreSQL database cluster dump
--

SET default_transaction_read_only = off;

SET client_encoding = 'UTF8';
SET standard_conforming_strings = on;

--
-- Roles
--

CREATE ROLE postgres;
ALTER ROLE postgres WITH SUPERUSER INHERIT CREATEROLE CREATEDB LOGIN REPLICATION;

--
-- PostgreSQL database cluster dump complete
--

postgres@tardis:~/dmp

\end{lstlisting}


\section{Backup validation}
\index{backup validation}
There's little advantage in having a backup if this is not restorable. The corruption danger is at 
various levels and unfortunately the problem appears only when the restore is needed.\newline 

The corruption have many causes. If the dump is saved locally a damaged filesystem can hide the 
problem meanwhile the corrupted block is not read. Also the disk subsystem with undetected problems 
will result in a silent corruption. In order to limit this kind of problems the filesystem of 
choice should be a solid one with strong journaling support.\newline

The disk subsystem should guarantee the data reliability rather the speed. Slow disks when backing 
up the data, in particular if in the compressed format don't limit the speed, being the cpu power 
the real bottleneck.\newline
If the dump file is transferred over the network is a good idea to generate a md5 checksum to 
check the file integrity after the transfer.\newline

All those measures don't give the security the backup is restorable. The only test capable to 
ensure the backup is good is a test restore on separate server. This can be a single test or a more 
structured check. Which strategy to adopt is determined by the amount of data, the time required 
for the restore and the backup schedule.\newline

The general purpose databases, which size is measurable in hundreds of gigabytes, the restore can 
complete in few hours and the continuous test is feasible. For the VLDB, which size is measured in 
terabytes, the restore can take more than one day, in particular if there are big indices requiring 
expensive sort on disk for the build. In this scenario a weekly restore gives a good perception if 
there are potential problems with the saved data. 
\chapter{Restore}
\label{cha:RESTORE}
There's little advantage in saving the data if the restore is not possible. In this chapter we'll take a 
look to the fastest and possibly the safest way to restore the saved dump.\newline

The program used for the restore is determined by the dump format. We'll first take a look to the restore 
using a plain format then the custom and  the directory formats. Finally we'll the way to improve 
the restore performances with a temporary sacrifice of the cluster's reliability.

\section{The plain format}
\label{sec:PLAINFORMAT}
As seen in \ref{cha:BACKUP} the pg\_dump's output is plain SQL. The generated script gives no choice but 
loading it into psql. The SQL statements are parsed and executed in sequence.\newline

This format have few advantages. For example it's possible to edit the statements using a common 
text editor. This of course if the dump is reasonably small. Even loading a file with vim when its size is 
measured in gigabytes becomes a stressful experience though.\newline

The data contents are saved using the COPY command. At restore time this choice have the best performance.

It's possible to save the data contents using the inserts. The restore is indeed very slow because each 
statement has to be parsed, planned and executed.\newline

If the backup saves the schema and the data in two separate files this requires extra care at dump time if 
there are triggers and foreign keys in the database schema. 

The data only backup should include the switch --disable-triggers which writes emit the DISABLE 
TRIGGER statements before the data load and the ENABLE TRIGGER after the data is restored.\newline

The following example shows a dump/reload session using the separate schema and data dump files.

Let's create a new database with a simple data structure. Two tables storing a city and the address 
and a foreign key between them enforcing the referential integrity.

\begin{lstlisting}[style=pgsql]
postgres=# CREATE DATABASE db_addr;
CREATE DATABASE
postgres=# \c db_addr 
You are now connected to database "db_addr" as user "postgres".
db_addr=# CREATE TABLE t_address
        (
                i_id_addr serial,
                i_id_city integer NOT NULL,
                t_addr text,
                CONSTRAINT pk_id_address PRIMARY KEY (i_id_addr)
        )
;
CREATE TABLE
db_addr=# CREATE TABLE t_city
        (
                i_id_city       serial,
                v_city          character varying (255),
                v_postcode      character varying (20),
                CONSTRAINT pk_i_id_city PRIMARY KEY (i_id_city)
        )
;
CREATE TABLE
db_addr=# ALTER TABLE t_address ADD 
	    CONSTRAINT fk_t_city_i_id_city FOREIGN KEY (i_id_city)  
	    REFERENCES t_city(i_id_city) 
	      ON DELETE CASCADE 
	      ON UPDATE RESTRICT;
ALTER TABLE

\end{lstlisting}

Now let's put some data into the tables.

\begin{lstlisting}[style=pgsql]
INSERT INTO t_city
	    ( 
	      v_city,	
	      v_postcode
	     )
      VALUES
	    (
	      'Leicester - Stoneygate',
	      'LE2 2BH'
	    )
      RETURNING i_id_city
;

 i_id_city 
-----------
         3
(1 row)

db_addr=# INSERT INTO t_address
	    ( 
	      i_id_city,	
	      t_addr
	     )
      VALUES
	    (
	      3,
	      '4, malvern road '
	    )
      RETURNING i_id_addr
;

 i_id_addr 
-----------
         1
(1 row)


\end{lstlisting}

We'll now execute dump the schema and the data in two separate plain files. Please note we are not using 
the --disable-triggers switch.

\begin{verbatim}
postgres@tardis:~/dmp$ pg_dump --schema-only db_addr > db_addr.schema.sql
postgres@tardis:~/dmp$ pg_dump --data-only db_addr > db_addr.data.sql

\end{verbatim}

Looking to the schema dump it's quite obvious what it does. All the DDL are saved in the correct 
order to restore the same database structure .\newline
The data is then saved by pg\_dump in the correct order for having the referential integrity 
guaranteed. In our very simple example the table t\_city is dumped before the table t\_address. 
This way the data will not violate the foreign key. In a complex scenario where multiple foreign keys are 
referring the same table, the referential order is not guaranteed. Let's run the same 
dump with the option --disable-trigger.

\begin{verbatim}
postgres@tardis:~/dmp$ pg_dump --disable-triggers --data-only db_addr > db_addr.data.sql

\end{verbatim}

The copy statements in this case are enclosed by two extra statements which disable and then re enable the 
triggers.

\begin{lstlisting}[style=pgsql]

ALTER TABLE t_address DISABLE TRIGGER ALL;

COPY t_address (i_id_addr, i_id_city, t_addr) FROM stdin;
1	3	4, malvern road 
\.


ALTER TABLE t_address ENABLE TRIGGER ALL;

\end{lstlisting}

The foreign keys and all the user defined trigger will not fire during the data restore, ensuring 
the data will be safely stored and improving the speed.\newline

Let's then create a new database where we'll restore the dump starting from the saved schema.

\begin{lstlisting}[style=pgsql]
postgres=# CREATE DATABASE db_addr_restore;
CREATE DATABASE
postgres=# \c db_addr_restore 
You are now connected to database "db_addr_restore" as user "postgres".
db_addr_restore=# \i db_addr.schema.sql 
SET
...
SET
CREATE EXTENSION
COMMENT
SET
SET
CREATE TABLE
ALTER TABLE
CREATE SEQUENCE
ALTER TABLE
ALTER SEQUENCE
CREATE TABLE
ALTER TABLE
CREATE SEQUENCE
ALTER TABLE
ALTER SEQUENCE
ALTER TABLE
...
ALTER TABLE
REVOKE
REVOKE
GRANT
GRANT
db_addr_restore=# \i db_addr.data.sql 
SET
...
SET
ALTER TABLE
...
ALTER TABLE
 setval 
--------
      1
(1 row)

 setval 
--------
      3
(1 row)

db_addr_restore=# \d
                   List of relations
 Schema |          Name           |   Type   |  Owner   
--------+-------------------------+----------+----------
 public | t_address               | table    | postgres
 public | t_address_i_id_addr_seq | sequence | postgres
 public | t_city                  | table    | postgres
 public | t_city_i_id_city_seq    | sequence | postgres
(4 rows)

\end{lstlisting}



\section{The binary formats}
\label{sec:PGDUMPBINFMT}
The three binary formats supported by pg\_dump are the custom, the directory and the tar format. 
The first two can be accessed randomly by the restore program and have the parallel restore 
support, being the best choice for having a flexible and reliable restore. Before the the 9.3 the 
only format supporting the parallel restore was the custom. With this version the directory 
format accepts the -j switch. This feature, combined with the parallel dump seen in 
\ref{sec:PGDUMPINT} is a massive improvement for saving big amount of data. The tar format does 
have the limit of 12 GB in the archive's file size and doesn't offer the parallel restore nor the 
selective restore. \newline

The custom format is a binary archive with a table of contents pointing the various archive 
sections. The directory format is a directory which name is the value provided with the -f switch. 
The directory contents are a toc.dat file, where the table of contents and the schema are stored. 
For each table there is a gzip file which name is a number corresponding to the toc entry for the 
saved relation. Those files store the data restore for the relation.\newline


The restore from the binary formats requires the pg\_restore usage. Because almost 
all the pg\_dump's switches are supported by pg\_restore we'll not repeat the look out. Take a look 
to  \ref{sec:PGDUMP} for the switch meanings. Anyway this is the pg\_restore's help output.
\newline 

\begin{verbatim}
pg_restore restores a PostgreSQL database from an archive created by pg_dump.

Usage:
  pg_restore [OPTION]... [FILE]

General options:
  -d, --dbname=NAME        connect to database name
  -f, --file=FILENAME      output file name
  -F, --format=c|d|t       backup file format (should be automatic)
  -l, --list               print summarized TOC of the archive
  -v, --verbose            verbose mode
  -V, --version            output version information, then exit
  -?, --help               show this help, then exit

Options controlling the restore:
  -a, --data-only              restore only the data, no schema
  -c, --clean                  clean (drop) database objects before recreating
  -C, --create                 create the target database
  -e, --exit-on-error          exit on error, default is to continue
  -I, --index=NAME             restore named index
  -j, --jobs=NUM               use this many parallel jobs to restore
  -L, --use-list=FILENAME      use table of contents from this file for
                               selecting/ordering output
  -n, --schema=NAME            restore only objects in this schema
  -O, --no-owner               skip restoration of object ownership
  -P, --function=NAME(args)    restore named function
  -s, --schema-only            restore only the schema, no data
  -S, --superuser=NAME         superuser user name to use for disabling triggers
  -t, --table=NAME             restore named table(s)
  -T, --trigger=NAME           restore named trigger
  -x, --no-privileges          skip restoration of access privileges (grant/revoke)
  -1, --single-transaction     restore as a single transaction
  --disable-triggers           disable triggers during data-only restore
  --no-data-for-failed-tables  do not restore data of tables that could not be
                               created
  --no-security-labels         do not restore security labels
  --no-tablespaces             do not restore tablespace assignments
  --section=SECTION            restore named section (pre-data, data, or post-data)
  --use-set-session-authorization
                               use SET SESSION AUTHORIZATION commands instead of
                               ALTER OWNER commands to set ownership

Connection options:
  -h, --host=HOSTNAME      database server host or socket directory
  -p, --port=PORT          database server port number
  -U, --username=NAME      connect as specified database user
  -w, --no-password        never prompt for password
  -W, --password           force password prompt (should happen automatically)
  --role=ROLENAME          do SET ROLE before restore

If no input file name is supplied, then standard input is used.

Report bugs to <pgsql-bugs@postgresql.org>.
 
\end{verbatim}


If the database connection is omitted pg\_restore sends the output to the standard output. The 
switch -f sends the output to a filename though. This is very useful if we want to check the 
original dump file is readable, executing a restore onto the /dev/null device.\newline
The the custom and directory formats show their power when restoring on a database 
connection in a multi core system. Using the -j switch it's possible to specify the number of 
parallel jobs for the data and the post data section. This can improve massively the 
recovery time, running the most time consuming actions in multiple processes.\newline

The word \textit{parallel} can be confusing in some way. PostgreSQL does not supports 
multithreading. That means each backend process will use just only one cpu. In this context, each 
job take care of a different area of the restore's table of contents, The TOC is split in 
many queues with a fixed object list to process. For example one queue will contain the data 
restoration for a table, and the relation's indices and constraints.\newline

The switch --section offers a fine grain control on which section of the archived data will be 
restored. In a custom and directory format there are three distinct sections. 
\begin{itemize}
 \item \textbf{pre-data} This section restores only the schema definitions not affecting the speed 
and reliability of the data restore. e.g. table's DDL, functions creation, extensions, etc.
\item  \textbf{data} The data restore itself, by default saved as COPY statements to speed up the 
process
\item  \textbf{post-data} This section runs the restore for all the objects enforcing the data 
integrity, like the primary and foreign keys, triggers and the indices which presence during the 
restore slow down the data reload massively.
\end{itemize}

The switch -C creates the target database before starting the restoration. To do this the 
connection must happen first on another database. \newline

We'll now will see how to restore the database seen in \ref{sec:PLAINFORMAT} in the same two 
steps approach, using the custom format.\newline

Let's start with a complete database dump using the custom format. 

\begin{verbatim}
postgres@tardis:~/dump$ pg_dump -Fc -f db_addr.dmp  db_addr
pg_dump: reading schemas
pg_dump: reading user-defined tables
pg_dump: reading extensions
pg_dump: reading user-defined functions
pg_dump: reading user-defined types
pg_dump: reading procedural languages
pg_dump: reading user-defined aggregate functions
pg_dump: reading user-defined operators
pg_dump: reading user-defined operator classes
pg_dump: reading user-defined operator families
pg_dump: reading user-defined text search parsers
pg_dump: reading user-defined text search templates
pg_dump: reading user-defined text search dictionaries
pg_dump: reading user-defined text search configurations
pg_dump: reading user-defined foreign-data wrappers
pg_dump: reading user-defined foreign servers
pg_dump: reading default privileges
pg_dump: reading user-defined collations
pg_dump: reading user-defined conversions
pg_dump: reading type casts
pg_dump: reading table inheritance information
pg_dump: reading event triggers
pg_dump: finding extension members
pg_dump: finding inheritance relationships
pg_dump: reading column info for interesting tables
pg_dump: finding the columns and types of table "t_address"
pg_dump: finding default expressions of table "t_address"
pg_dump: finding the columns and types of table "t_city"
pg_dump: finding default expressions of table "t_city"
pg_dump: flagging inherited columns in subtables
pg_dump: reading indexes
pg_dump: reading indexes for table "t_address"
pg_dump: reading indexes for table "t_city"
pg_dump: reading constraints
pg_dump: reading foreign key constraints for table "t_address"
pg_dump: reading foreign key constraints for table "t_city"
pg_dump: reading triggers
pg_dump: reading triggers for table "t_address"
pg_dump: reading triggers for table "t_city"
pg_dump: reading rewrite rules
pg_dump: reading large objects
pg_dump: reading dependency data
pg_dump: saving encoding = UTF8
pg_dump: saving standard_conforming_strings = on
pg_dump: saving database definition
pg_dump: dumping contents of table t_address
pg_dump: dumping contents of table t_city

\end{verbatim}

We'll use a second database for the restore.

\begin{lstlisting}[style=pgsql]
postgres=# CREATE DATABASE db_addr_restore_bin;
CREATE DATABASE

\end{lstlisting}

We'll then restore just the schema using the following command.

\begin{verbatim}
postgres@tardis:~/dump$ pg_restore -v -s -d db_addr_restore_bin db_addr.dmp 
pg_restore: connecting to database for restore
pg_restore: creating SCHEMA public
pg_restore: creating COMMENT SCHEMA public
pg_restore: creating EXTENSION plpgsql
pg_restore: creating COMMENT EXTENSION plpgsql
pg_restore: creating TABLE t_address
pg_restore: creating SEQUENCE t_address_i_id_addr_seq
pg_restore: creating SEQUENCE OWNED BY t_address_i_id_addr_seq
pg_restore: creating TABLE t_city
pg_restore: creating SEQUENCE t_city_i_id_city_seq
pg_restore: creating SEQUENCE OWNED BY t_city_i_id_city_seq
pg_restore: creating DEFAULT i_id_addr
pg_restore: creating DEFAULT i_id_city
pg_restore: creating CONSTRAINT pk_i_id_city
pg_restore: creating CONSTRAINT pk_id_address
pg_restore: creating FK CONSTRAINT fk_t_city_i_id_city
pg_restore: setting owner and privileges for DATABASE db_addr
pg_restore: setting owner and privileges for SCHEMA public
pg_restore: setting owner and privileges for COMMENT SCHEMA public
pg_restore: setting owner and privileges for ACL public
pg_restore: setting owner and privileges for EXTENSION plpgsql
pg_restore: setting owner and privileges for COMMENT EXTENSION plpgsql
pg_restore: setting owner and privileges for TABLE t_address
pg_restore: setting owner and privileges for SEQUENCE t_address_i_id_addr_seq
pg_restore: setting owner and privileges for SEQUENCE OWNED BY t_address_i_id_addr_seq
pg_restore: setting owner and privileges for TABLE t_city
pg_restore: setting owner and privileges for SEQUENCE t_city_i_id_city_seq
pg_restore: setting owner and privileges for SEQUENCE OWNED BY t_city_i_id_city_seq
pg_restore: setting owner and privileges for DEFAULT i_id_addr
pg_restore: setting owner and privileges for DEFAULT i_id_city
pg_restore: setting owner and privileges for CONSTRAINT pk_i_id_city
pg_restore: setting owner and privileges for CONSTRAINT pk_id_address
pg_restore: setting owner and privileges for FK CONSTRAINT fk_t_city_i_id_city

\end{verbatim}

The dump file is specified as last parameter. The -d switch tells pg\_restore which database to 
connect for the restore. By default the postgres user usually connects using the ident 
operating system daemon or the trust authentication method, when connected as local. That's the 
reason why in this example there's no need of specifying the username or enter the 
password.\newline 

The second restore's step is the data load. In the example seen in \ref{sec:PLAINFORMAT} we used 
the pg\_dump with --disable-triggers switch in order to avoid failures caused by constraint 
violation. With the custom format the switch is used at restore time.

\begin{verbatim}
postgres@tardis:~/dump$ pg_restore --disable-triggers -v -a -d db_addr_restore_bin db_addr.dmp 
pg_restore: connecting to database for restore
pg_restore: disabling triggers for t_address
pg_restore: processing data for table "t_address"
pg_restore: enabling triggers for t_address
pg_restore: executing SEQUENCE SET t_address_i_id_addr_seq
pg_restore: disabling triggers for t_city
pg_restore: processing data for table "t_city"
pg_restore: enabling triggers for t_city
pg_restore: executing SEQUENCE SET t_city_i_id_city_seq
pg_restore: setting owner and privileges for TABLE DATA t_address
pg_restore: setting owner and privileges for SEQUENCE SET t_address_i_id_addr_seq
pg_restore: setting owner and privileges for TABLE DATA t_city
pg_restore: setting owner and privileges for SEQUENCE SET t_city_i_id_city_seq
 
\end{verbatim}

However, this approach does not prevent the slowness caused by the indices when reloading the data.
If a restore with multiple steps is required (e.g. creating the database schema and check all the 
relations are in place before starting) the section switch is a better choice. Let's see how it 
works with the example seen before.\newline

We'll first restore the pre-data section\footnote{Don't forget to clear the existing objects in 
the database.}.

\begin{verbatim}
postgres@tardis:~/dump$ pg_restore --section=pre-data -v  -d db_addr_restore_bin db_addr.dmp 
pg_restore: connecting to database for restore
pg_restore: creating SCHEMA public
pg_restore: creating COMMENT SCHEMA public
pg_restore: creating EXTENSION plpgsql
pg_restore: creating COMMENT EXTENSION plpgsql
pg_restore: creating TABLE t_address
pg_restore: creating SEQUENCE t_address_i_id_addr_seq
pg_restore: creating SEQUENCE OWNED BY t_address_i_id_addr_seq
pg_restore: creating TABLE t_city
pg_restore: creating SEQUENCE t_city_i_id_city_seq
pg_restore: creating SEQUENCE OWNED BY t_city_i_id_city_seq
pg_restore: creating DEFAULT i_id_addr
pg_restore: creating DEFAULT i_id_city
pg_restore: setting owner and privileges for DATABASE db_addr
pg_restore: setting owner and privileges for SCHEMA public
pg_restore: setting owner and privileges for COMMENT SCHEMA public
pg_restore: setting owner and privileges for ACL public
pg_restore: setting owner and privileges for EXTENSION plpgsql
pg_restore: setting owner and privileges for COMMENT EXTENSION plpgsql
pg_restore: setting owner and privileges for TABLE t_address
pg_restore: setting owner and privileges for SEQUENCE t_address_i_id_addr_seq
pg_restore: setting owner and privileges for SEQUENCE OWNED BY t_address_i_id_addr_seq
pg_restore: setting owner and privileges for TABLE t_city
pg_restore: setting owner and privileges for SEQUENCE t_city_i_id_city_seq
pg_restore: setting owner and privileges for SEQUENCE OWNED BY t_city_i_id_city_seq
pg_restore: setting owner and privileges for DEFAULT i_id_addr
pg_restore: setting owner and privileges for DEFAULT i_id_city
 
\end{verbatim}


Again the pg\_restore loads the objects with the ownership and privileges. What's missing is the 
constraints creation. The second step is the data section's load.

\begin{verbatim}
postgres@tardis:~/dump$ pg_restore --section=data -v  -d db_addr_restore_bin db_addr.dmp 
pg_restore: connecting to database for restore
pg_restore: implied data-only restore
pg_restore: processing data for table "t_address"
pg_restore: executing SEQUENCE SET t_address_i_id_addr_seq
pg_restore: processing data for table "t_city"
pg_restore: executing SEQUENCE SET t_city_i_id_city_seq
pg_restore: setting owner and privileges for TABLE DATA t_address
pg_restore: setting owner and privileges for SEQUENCE SET t_address_i_id_addr_seq
pg_restore: setting owner and privileges for TABLE DATA t_city
pg_restore: setting owner and privileges for SEQUENCE SET t_city_i_id_city_seq

\end{verbatim}

This section simply loads the table's data and sets the sequence values. Apart for the ownership no 
further action is performed. Finally we'll run the post-data section.

\begin{verbatim}
postgres@tardis:~/dump$ pg_restore --section=post-data -v  -d db_addr_restore_bin db_addr.dmp 
pg_restore: connecting to database for restore
pg_restore: creating CONSTRAINT pk_i_id_city
pg_restore: creating CONSTRAINT pk_id_address
pg_restore: creating FK CONSTRAINT fk_t_city_i_id_city
pg_restore: setting owner and privileges for CONSTRAINT pk_i_id_city
pg_restore: setting owner and privileges for CONSTRAINT pk_id_address
pg_restore: setting owner and privileges for FK CONSTRAINT fk_t_city_i_id_city

\end{verbatim}

With this run the constrains (and eventually all the indices) are created in the best 
approach possible when dealing with the bulk data processing. \newline

Loading the data contents without indices maximise the speed. The constraint and index build with 
the data already in place results in a faster build and a fresh index without any bloat.

\section{Restore performances}

When restoring a database, in particular in a disaster recovery scenario, the main goal is 
to have the data back on line as fast as possible. Usually the data section's restore 
is fast. If the dump has been taken using the copy statements, which are enabled by default, 
the reload requires a fraction of the entire restore's time. Taking the advantage of the parallel 
jobs, available for the custom and directory format, it's still possible to improve the data 
section's reload.\newline

The other face of the coin is the post-data section. Because the objects in this section are mostly 
random access operations, the completion can require more time than the data section itself; even 
if the size of the resulting objects is smaller than the table's data. This happens because the 
unavoidable sort operations are CPU and memory bound. The parallel restore gives some advantage, 
but as seen in \ref{sec:PGDUMPBINFMT} each loop's process is single threaded. \newline

Setting up an emergency postgresql.conf file can speed up the restore, reducing the time up to 40\% 
than the production's configuration. What it follows assumes the production's database is lost and 
the data is restored from a custom format's backup. 

\subsection{shared\_buffers}
When reloading the data from the dump, the database performs a so called bulk load operation. The 
PostgreSQL's memory manager have a subroutine which protects the shared segment from the block 
eviction caused by IO intensive operations. It's then very likely the ring buffer strategy will be 
triggered by the restore, sticking the IO in a small 4 MB buffer protecting the rest of the memory. 
A big shared buffer it can cache the data pages when in production but becomes useless when 
restoring. A smaller shared buffer, enough to hold the IO from  the restore processes will result 
in more memory available for the backends when processing the post-data section. There's no fixed 
rule for the sizing. A gross approximation could be 10 MB for each parallel job with a minimum cap 
of 512 MB.   

\subsection{wal\_level}
The wal\_level parameter sets the level of redo informations stored in the WAL segments. By default 
is set to minimal, enabling the xlog skip. Having the database in with a standby, or simply using 
the point in time recovery as alternate backup strategy requires the parameter to be set to archive 
or hot\_standby. If this is the case and you have a PITR or standby to failover, stop reading 
this book and act immediately. Restoring from a physical backup is several time faster than a 
logical restore. If you have lost the standby or PITR snapshot then before starting the reload the 
wal\_level must be set to minimal.

\subsection{fsync}
Turning off fsync can improve massively the restore's speed. Having this parameter turned off
is not safe, unless the cache is have the backup battery to prevent data loss in case of power
failure. However, even without the battery at restore time having the fsync off is not critical.
After all the database is lost, what else can happen?

\subsection{checkpoint\_segments, checkpoint\_timeout}
The checkpoint is a vital event in the database activity. When occurs all the pages not yet written
to the data files are synced to disk. This in the restore context is a disturbance. Increasing the
checkpoint segments and the timeout to the maximum allowed values will avoid any extra IO. In any
case the dirtied blocks will be written on disk when the buffer manager will need to free some
space.


\subsection{autovacuum}
There's no point in having vacuumed the tables after a complete reload. Unfortunately autovacuum
does not know if a table is being restored. When the limit for the updated tuples is recognised the
daemon starts a new process wasting precious CPU cycles. Turning off temporarily the setting will
let the backends to stay focused on the main goal. The data restore.


\subsection{max\_connections}
Limiting the max connections to number of restore jobs is a good idea. It's ok also giving a slight
headroom for one or two connections, just in case there's need to log in and check the database
status. This way the available memory can be shared efficiently between the backends.

\subsection{maintenance\_work\_memory}
This parameter affects the index builds which are stored in the restore's post-data section. Low 
values will results in the backends sorting on disk and slowing down the entire process. Higher
values will keep the index build in memory with great speed gain. The value should be carefully
sized keeping in mind the memory available on the system. This value should be reduced by a 20\% if
the total ram is up to 10 GB and by 10\% if bigger. This reduction is needed to consider the memory
consumed by the operating system and the other processes. From the remaining ram must be subtracted
the shared\_buffer's memory. The remaining value must be divided by the expected backends to
perform the restore. For example if we have a system with 26GB a shared\_buffer of 2 GB and 10
parallel jobs to execute the restore, the maintenance\_work\_mem is  2.14 GB.
\newpage
\begin{verbatim}
26 - 10% =  23.4
23.4 - 2 = 21.4
21.4 / 10 = 2.14
\end{verbatim}

Ignoring this recommendation can trigger the swap usage resulting in a slower restore process.

%\chapter{A couple of things to know before start coding...}
%\label{cha:COUPLETHINGS}
\chapter{A couple of things to know before start coding...}
\label{cha:COUPLETHINGS}


\appendix
\input{chapters/License}
\listoffigures
\listoftables
\printindex{}
\end{document}
