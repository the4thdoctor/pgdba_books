\chapter{Nuts and bolts}
Before to get knee deep with the book's topic I want to explain how to set up an
efficient environment and a set of methods which can improve the quality of coding.
Some of those things are in open contrast with the general code style trending.
Unfortunately the SQL universe is a strange place and requires strange behaviours.
There is a visual match between the query's sections and the logic which requires a clear
and well defined formatting to make it understandable.\newline
Let's start then with a todo list for our workshop.
\section{The editor}
Unlikely many commercial RDBMS PostgreSQL, alongside his command line psql, have many unofficial
clients. Those products have generally a good support for the database features and  a good
connectivity layer. There is an exhaustive list on the PostgreSQL wiki\newline
\href{https://wiki.postgresql.org/wiki/Community\_Guide\_to\_PostgreSQL\_GUI\_Tools}{
https://wiki.postgresql.org/wiki/Community\_Guide\_to\_PostgreSQL\_GUI\_Tools}. I'm not telling you
which editor is the best. I tested some of them, TOra, SQL workbench and SQL Maestro. I finally
decided to use PgAdmin 3 because it offered me the best compromise between editing and database
schema browsing.\newline

The tool of choice should have the following features.

\subsection{Native connector}
One of the reasons I did not considered SQL workbench it was the JDBC connector. For me writing and
testing the SQL is a process similar to the compulsive obsessive disorder. I write, I test, I
change the query I test again. In this context the quick reaction from the client is absolutely
important. The native connector have virtually no lag, except the disk/network bandwidth.

