\documentclass[oneside]{book}
\usepackage[utf8]{inputenc}
\usepackage{makeidx}
\usepackage{graphicx}
\setcounter{tocdepth}{2}
\usepackage{hyperref}
\hypersetup{pdfborder={0 0 0},colorlinks=false}
\usepackage{verbatim}
\newenvironment{smallverbatim}{\endgraf\small\verbatim}{\endverbatim}
\newenvironment{tinyverbatim}{\endgraf\scriptsize\verbatim}{\endverbatim}
\pagestyle{plain}
\usepackage{float}
\usepackage{pdfpages}
\usepackage{listings}
\usepackage[paperwidth=21.59cm,paperheight=27.94cm]{geometry}
\usepackage{xcolor}

\author{Federico Campoli}
\title{PostgreSQL Database Administration \\ Volume 2 \\ Architecture}


\makeindex

\lstdefinestyle{pgsql}{
  belowcaptionskip=1\baselineskip,
  breaklines=true,
  frame=l,
  language=SQL,
  showstringspaces=false,
  basicstyle=\footnotesize\ttfamily,
  keywordstyle=\bfseries\color{green!40!black},
  commentstyle=\itshape\color{purple!40!black},
  identifierstyle=\color{blue},
  stringstyle=\color{orange},
  morekeywords={VACUUM, FULL, ANALYZE, TABLESPACE,SET}
}


\begin{document}
\includepdf{covers/cover_good.pdf}
\maketitle

\newpage{}



\tableofcontents{}

\chapter*{Preface}
Here we go, again. 




\section*{Intended audience}
Database administrators, System administrators

\section*{Book structure}
In this book starts where the PostgreSQL DBA book 1 ends.\newline


\section*{Version and platform}
This book cover the database version 9.3 on Debian GNU Linux 7.0.
References to older version or different platform are explicitly specified.

\chapter{PostgreSQL's Architecture}
A PostgreSQL cluster is a directory initialised by the program initdb plus the eventual tablespaces configured. 
When the  postgres process is started the files composing the cluster are accessed in the startup process and, when this is complete
the system is made accessible for usage.




\chapter{The memory}
\label{ch:PGMEMORY}
\input{chapters/Data_area}

\chapter{The cluster up and running}
\chapter{Performance Tuning}
\chapter{Point in time recovery}
\chapter{High availability}

\appendix

\listoffigures
\listoftables
\printindex{}
\end{document}
