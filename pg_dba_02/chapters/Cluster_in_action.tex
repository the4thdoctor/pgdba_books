\chapter{The cluster in action}
PostgreSQL delivers the data ensuring the ACID is enforced at any time. This chapter will give an outlook 
on a ``day in the life'' of a PostgreSQL's cluster.

\section{After the startup}
When the cluster completes the startup procedure it starts accepting the connections. When a connection  
is successful then the postgres main process forks into a new backend process which is assigned to the 
connection for the client's lifetime. The fork is quite expensive and does not work very well for a high 
rate of connection's requests. The maximum number of connections is set at startup and cannot be changed 
dynamically. Whether the connection is used or not for each connection slot are consumed ~400 bytes of 
shared memory.\newline

Even without connections the cluster keeps himself busy. 

\section{The write ahead log}
The \index{write ahead log} data pages are stored into the shared buffer for read or for write. A mechanism 
called pinning ensures that only one backend at time is accessing the page. If the page\index{page, dirty} 
is modified the becomes dirty, which 
means the page is not yet written on the data file. The page's change is first saved on the write ahead log 
as WAL record. The commit status for the transactions is then stored in the directory clog or the directory
pg\_serial, depending on the transaction isolation leve. Those information is all the cluster needs to 
rebuild his last consistent state if an unclean shutdown happens. The writes on the WAL are managed by a 
background process called WAL writer\index{WAL writer}. This process were first introduced with PostgreSQL 
8.3. The wal records are stored into a shared buffer's area sized by the parameter wal\_buffers before the 
flush on disk. The WAL records are stored on fixed length segments. When the cluster fill up a WAL segment 
and starts a new one there is a xlog switch.

\section{The checkpoint}
On a regular basis the cluster starts a fundamental activity called checkpoint. The frequency of this 
action is governed by the time between two distinct checkpoints and the space, measured in log 
switches between two checkpoints. The checkpoint scans the shared buffer and writes down to the data 
files all the dirty pages he find. When the checkpoint is complete the process determines the last 
checkpoint location on the WAL segments and write down this important information on the control file 
stored into the cluster's pg\_global tablespace.


\section{The background writer}


\section{Transactions}


\section{The backends}

