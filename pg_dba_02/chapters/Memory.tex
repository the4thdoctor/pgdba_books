\chapter{The memory}
\label{ch:PGMEMORY}
The PostgreSQL memory at first sight looks simple. If compared with the complex structures implemented in 
the other DBMS to a careless reader could seem rudimentary. However, the memory and in particular the 
shared buffers implementation is complex and sophisticated. This chapter will dig down deep into the 
PostgreSQL's memory.

\section{The shared buffer}
The shared buffer is a segment allocated at cluster's startup. Its size is determined by the GUC parameter 
shared\_buffers and the size can be changed only restarting the cluster. The shared buffer is used 
to manage the data pages as seen in \ref{sec:CLUBACKEND}, which are called buffers when loaded in 
rhw memory. Having a RAM segment is not uncommon in the database universe. Is a rapid exchange 
area where the buffers are kept across the sessions and keeps the data near the CPU. Like his competitors 
PostgreSQL strives to keep in memory what is important things and not everything. In the era of the 
\textit{in memory databases} this could seems an obsolete concept. Unfortunately the truth is that the 
resources, and so the money, are not infinite. Before digging into the technical details we'll have a look 
to the history of the memory manager. The way PostgreSQL sticks an elephant into a small car without 
Timelord technology.

Back in the old days PostgreSQL was quite rudimentary. The version 7.4 did not had tablespaces, it was 
without a mechanism to prevent the XID wraparound failure (except for the routine vacuuming of course) 
and the free space manager was a simple LRU buffer. Since then the algorithm evolved and became a 
very efficient system to cache effectively the buffers. We'll look briefly to the free space manager's 
history. 

\subsection{PostgreSQL 7.4, the LRU list}
In PostgreSQL 7.4 the free space in the shared buffer was managed using a simple last recently used list. 
When a page is first request from disk the buffer manager puts it into the first buffer in the free LRU 
list. If the first buffer is used by another page the list is shifted by one unity and the eventual last 
page in the list is discarded. If a page in the list is requested again is put back into the first buffer 
and so on.

\subsubsection{The tale of the buffer}

\subsection{PostgreSQL 8.0, the 2q}
\subsubsection{the ARC}

\subsection{PostgreSQL 8.1, the clock sweep}

\subsection{PostgreSQL 9.3, SHMMAX no more}
%todo, the change of memory allocation method happened with the 9.3

\subsection{PostgreSQL 9.4, huge pages}


\subsection{The wal buffers}

\subsection{The memory manager}
\section{The user memory}
\subsection{Work memory}
\subsection{Maintenance work memory}
\subsection{Temporary memory}


\subsection{Memory context}

\section{Wrap up}